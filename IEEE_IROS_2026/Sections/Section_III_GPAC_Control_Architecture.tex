This section presents the \emph{Geometric Position and Attitude Control} (GPAC) framework, a four-layer hierarchical controller for cooperative cable-suspended transport. The architecture exploits the geometric structure of the configuration manifold~\eqref{eq:config_manifold} by assigning each physical degree of freedom to the appropriate control layer, operating at progressively higher bandwidths from position tracking through disturbance rejection. Fig.~\ref{fig:gpac_architecture} illustrates the information flow.

\begin{figure}[t]
\centering
\begin{tikzpicture}[
  node distance=0.5cm and 0.6cm,
  % Block styles
  ctrl/.style={draw=blue!60, fill=blue!7, thick, rounded corners=3pt,
    minimum height=0.78cm, minimum width=3.2cm,
    align=center, font=\footnotesize, inner sep=3pt},
  safe/.style={draw=red!55, fill=red!7, thick, rounded corners=3pt,
    minimum height=0.78cm, minimum width=3.2cm,
    align=center, font=\footnotesize, inner sep=3pt},
  est/.style={draw=teal!60, fill=teal!7, thick, rounded corners=3pt,
    minimum height=0.78cm, minimum width=2.3cm,
    align=center, font=\footnotesize, inner sep=3pt},
  hw/.style={draw=black!45, fill=black!4, thick, rounded corners=3pt,
    minimum height=0.78cm,
    align=center, font=\footnotesize, inner sep=3pt},
  arr/.style={-{Stealth[length=3.5pt, width=2.8pt]}, semithick, black!75},
  darr/.style={-{Stealth[length=3.5pt, width=2.8pt]}, semithick,
    densely dashed, teal!70!black},
  lbl/.style={font=\scriptsize, fill=white, inner sep=1pt},
  bw/.style={font=\sffamily\tiny, text=black!40},
]
% ====== LEFT COLUMN: CONTROL CASCADE ======
\node[ctrl] (L1) {\textbf{Layer\,1}\; Position $+$ Anti-Swing};
\node[safe, below=of L1] (CBF)
  {\textbf{CBF Safety Filter}\\[-2pt]
   {\scriptsize $h_T\!\cdot\!h_\theta\!\cdot\!h_\omega\!\cdot\!h_\phi\!\cdot\!h_{\mathrm{col}}$}};
\node[ctrl, below=of CBF] (L2)
  {\textbf{Layer\,2}\; Attitude on $\SOthree$};
% ====== RIGHT COLUMN: ESTIMATION ======
\node[est, right=of L1] (L3) {\textbf{Layer\,3}\\[-2pt] CL Estimator};
\node[est, right=of CBF] (L4) {\textbf{Layer\,4}\\[-2pt] ESO};
\node[est, right=of L2] (ESKF) {\textbf{ESKF}\\[-2pt] Sensor Fusion};
% ====== BOTTOM: PLANT ======
\node[hw, below=0.6cm of $(L2.south)!0.5!(ESKF.south)$,
  minimum width=6.6cm] (Plant)
  {Quadrotor\,$i$ \;$+$\; Bead-Chain Cable \;$+$\; Payload};
% ====== BANDWIDTH LABELS ======
\node[bw, anchor=south east] at (L1.north east) {50\,Hz};
\node[bw, anchor=south east] at (CBF.north east) {200\,Hz};
\node[bw, anchor=south east] at (L2.north east) {200\,Hz};
\node[bw, anchor=south west] at (L3.north west) {200\,Hz};
\node[bw, anchor=south west] at (L4.north west) {$\omega_o\!=\!50$};
\node[bw, anchor=south west] at (ESKF.north west) {10--200\,Hz};
% ====== REFERENCE INPUT ======
\node[above=0.3cm of L1, font=\footnotesize] (ref)
  {$p_d(t),\; v_d(t),\; a_d(t)$};
\draw[arr] (ref) -- (L1);
% ====== CONTROL CASCADE ARROWS ======
\draw[arr] (L1) -- node[lbl, right] {$F_{\!\mathrm{des}},\; R_d$} (CBF);
\draw[arr] (CBF) --
  node[lbl, right] {$f_{\mathrm{safe}},\; R_d^{\mathrm{safe}}$} (L2);
\draw[arr] (L2.south) --
  node[lbl, left, pos=0.4] {$\tau_i,\; f_i$} (L2.south |- Plant.north);
% ====== PLANT TO ESTIMATION ======
\draw[arr] (ESKF.south |- Plant.north) --
  node[lbl, right, pos=0.4] {\scriptsize sensors} (ESKF.south);
% ====== ESTIMATION CASCADE ======
\draw[arr] (ESKF) --
  node[lbl, right] {$\hat{p}_i, \hat{v}_i$} (L4);
\draw[arr] (L4) -- (L3);
% Cable state from plant to L3
\draw[arr, black!55] (Plant.east) -- ++(0.15,0) |-
  node[lbl, right, pos=0.25] {$T_i, q_i$} (L3.east);
% ====== FEEDBACK TO LAYER 1 ======
\draw[arr, blue!60] (L3.west) --
  node[lbl, above] {$\hat{\theta}_i$} ([yshift=3pt]L1.east);
\draw[darr] (L4.west) -- ++(-.2, 0) |-
  node[lbl, below, pos=0.78] {$\hat{d}_i$}
  ([yshift=-3pt]L1.east);
% ====== PER-AGENT BOUNDARY ======
\node[draw=black!20, densely dashed, rounded corners=5pt,
  fit=(ref)(L1)(L2)(L3)(ESKF)(Plant),
  inner xsep=8pt, inner ysep=6pt,
  label={[font=\scriptsize\itshape, text=black!35,
    anchor=south east]south east:per agent\,$i$}] {};
\end{tikzpicture}%
\caption{GPAC hierarchical architecture for a single agent. \textbf{Left column} (blue): control cascade---Layer~1 computes the desired force $F_{\mathrm{des}}$ and extracts $R_d$; the CBF safety filter (red) enforces tautness, angle, tilt, and collision constraints; Layer~2 tracks the safe attitude on $\SOthree$. \textbf{Right column} (teal): estimation cascade---the ESKF fuses IMU/GPS/barometer; Layer~4 (ESO, continuous-time, $\omega_o = 50$\,rad/s) estimates lumped disturbances $\hat{d}_i$; Layer~3 (concurrent learning, 50\,Hz) estimates the payload mass share $\hat{\theta}_i \to m_L/N$. Dashed arrows denote feedforward paths. Each agent executes this pipeline independently.}
\label{fig:gpac_architecture}
\end{figure}

\paragraph{Information pattern} Each agent~$i$ receives at runtime: (i)~a shared reference trajectory $p_d(t)$ (broadcast once before flight or via a low-bandwidth downlink); (ii)~its own IMU, GPS, and barometer measurements; (iii)~local cable tension $T_i$ and direction $q_i$ from attachment-point sensors (Section~\ref{sec:modeling:cable_sensing}). For the collision avoidance barrier $h_{\mathrm{col},ij}$, neighbor positions $p_j$ are obtained via GPS broadcast (each drone transmits its own position at the GPS rate of 10\,Hz). No payload state, no other agent's cable state, and no centralized coordinator are required.

% ============================================================
\subsection{Layer 1: Position Tracking and Anti-Swing Control}
\label{sec:control:layer1}
% ============================================================

The outermost layer computes a desired total thrust vector $F_{\mathrm{des},i} \in \R^3$ for each quadrotor by combining trajectory tracking, cable tension compensation, anti-swing regulation on $\Sph^2$, and disturbance feedforward. This layer operates at the effective position-control bandwidth of approximately 50\,Hz.

\subsubsection{Trajectory Tracking}

Let $p_{d_i}(t)$, $v_{d_i}(t)$, and $a_{d_i}(t)$ denote the desired position, velocity, and acceleration for quadrotor $i$, generated from a shared waypoint trajectory with individual formation offsets (Section~\ref{sec:modeling:simulation}). Define the tracking errors $e_{p_i} = p_i - p_{d_i}$ and $e_{v_i} = v_i - v_{d_i}$, and the integral error $e_{I_i} = \int_0^t e_{p_i}(\tau)\,d\tau$ with per-axis anti-windup clamping $\abs{e_{I_i,k}} \leq \bar{e}_I = 2.0$. The PID feedback force is
\begin{equation}
  F_{\mathrm{fb},i} = -K_p\,e_{p_i} - K_d\,e_{v_i} - K_i\,e_{I_i},
  \label{eq:layer1_pid}
\end{equation}
where $K_p = \diag(6, 6, 8)$, $K_d = \diag(8, 8, 10)$, and $K_i = \diag(0.1, 0.1, 0.2)$ are diagonal gain matrices. The feedforward force compensates gravity and the nominal trajectory acceleration:
\begin{equation}
  F_{\mathrm{ff},i} = m_Q\,(a_{d_i} + g\,e_3).
  \label{eq:layer1_ff}
\end{equation}

\subsubsection{Cable Tension Compensation}

During cooperative transport, the cable exerts a downward force on each quadrotor. To prevent the position controller from treating this force as a disturbance, we add a tension feedforward term:
\begin{equation}
  F_{\mathrm{cable},i} = \kappa_T\,T_i\,q_i,
  \label{eq:tension_ff}
\end{equation}
where $T_i$ is the measured top-segment cable tension, $q_i \in \Sph^2$ is the measured cable direction (pointing from the payload toward the quadrotor), and $\kappa_T = 1.0$ is the compensation gain. During the pickup phase, a ramped tension target prevents impulsive loading: the target tension increases linearly from $0$ to $T_{\mathrm{target}} = m_L g / N$ over a duration $t_{\mathrm{ramp}} = 2.0$\,s after the cable first becomes taut (detected when $T_i \geq T_{\mathrm{detect}} = 1.0$\,N). A proportional feedback term $K_{T_p}(T_{\mathrm{target}} - T_i)$ with $K_{T_p} = 0.5$ and an altitude adjustment $\Delta z_i = K_{z}\,(T_{\mathrm{target}} - T_i)$ with $K_z = 0.003$\,m/N (clamped to $\pm 0.5$\,m) smooth the transition.

\subsubsection{Anti-Swing Control on $\Sph^2$}

Pendular oscillations of the cable-suspended payload degrade tracking performance and can excite resonances in the bead-chain model. We suppress these oscillations using a control law defined intrinsically on $\Sph^2$~\cite{lee2018geometric}. Let $q_{d_i} \in \Sph^2$ be the desired cable direction (typically $-e_3$, i.e., hanging vertically downward). The cable direction error $e_{q_i}$~\eqref{eq:eq_model} lies in the tangent space at $q_i$ and vanishes if and only if $q_i = q_{d_i}$. Let $\omega_{q_i}$ denote the cable angular velocity projected onto $T_{q_i}\Sph^2$ via $\omega_{q_i} \leftarrow P(q_i)\,\omega_{q_i}$. The anti-swing control force is
\begin{equation}
  F_{\mathrm{swing},i} = k_q\,e_{q_i} - k_\omega\,\omega_{q_i},
  \label{eq:antiswing_law}
\end{equation}
with $k_q = 4.0$ and $k_\omega = 2.0$. The proportional term steers $q_i$ toward $q_{d_i}$ along the geodesic on $\Sph^2$, and the derivative term damps the cable swing rate. Together, these terms reduce the configuration error $\Psi_{q_i} = 1 - q_{d_i}^\top q_i$ monotonically when no external perturbation is present.

\subsubsection{Disturbance Feedforward}

The Extended State Observer (Layer~4, Section~\ref{sec:control:layer4}) provides an estimate $\hat{d}_i \in \R^3$ of the lumped disturbance acceleration acting on quadrotor $i$. This estimate is fed forward as
\begin{equation}
  F_{\mathrm{eso},i} = m_Q\,\hat{d}_i.
  \label{eq:eso_ff}
\end{equation}

\subsubsection{Total Thrust Vector and Desired Rotation}

The total desired thrust vector is the sum of all contributions:
\begin{equation}
  F_{\mathrm{des},i} = F_{\mathrm{ff},i} + F_{\mathrm{fb},i} + F_{\mathrm{cable},i} + F_{\mathrm{swing},i} + F_{\mathrm{eso},i},
  \label{eq:total_thrust}
\end{equation}
subject to saturation $\norm{F_{\mathrm{des},i}} \in [f_{\min}, f_{\max}]$ with $f_{\min} = 0$ and $f_{\max} = 150$\,N. The scalar thrust magnitude is $f_i = \norm{F_{\mathrm{des},i}}$.

The desired rotation matrix $R_{d_i} \in \SOthree$ is constructed from $F_{\mathrm{des},i}$ and a desired yaw angle $\psi_{d_i}$ following~\cite{lee2010geometric}:
\begin{equation}
  b_{3_c} = \frac{F_{\mathrm{des},i}}{\norm{F_{\mathrm{des},i}}}, \quad
  b_{1_d} = \begin{bmatrix} \cos\psi_{d_i} \\ \sin\psi_{d_i} \\ 0 \end{bmatrix}\!,
  \label{eq:b3c_b1d}
\end{equation}
\begin{equation}
  b_{2_c} = \frac{b_{3_c} \times b_{1_d}}{\norm{b_{3_c} \times b_{1_d}}}, \qquad
  b_{1_c} = b_{2_c} \times b_{3_c},
  \label{eq:desired_rotation}
\end{equation}
\begin{equation}
  R_{d_i} = \begin{bmatrix} b_{1_c} & b_{2_c} & b_{3_c} \end{bmatrix}\!.
  \label{eq:Rd_assembled}
\end{equation}
This construction ensures that the body $z$-axis aligns with the desired thrust direction and the heading aligns with $\psi_{d_i}$ as closely as possible, producing a rotation matrix that is always well-defined except when $b_{3_c}$ is parallel to $b_{1_d}$ (handled by a fallback in the implementation).

% ============================================================
\subsection{Layer 2: Geometric Attitude Control on $\SOthree$}
\label{sec:control:layer2}
% ============================================================

Given the desired rotation $R_{d_i}$ from Layer~1, the inner attitude loop computes body-frame control torques using the geometric tracking controller of Lee~et~al.~\cite{lee2010geometric}. This layer operates at 200\,Hz and provides almost-global exponential stability on $\SOthree$ without the kinematic singularities inherent in Euler-angle representations.

\subsubsection{Error Functions on $\SOthree$}

The attitude error vector $e_{R_i}$~\eqref{eq:eR_model} and the angular velocity error
\begin{equation}
  e_{\Omega_i} = \Omega_i - R_i^\top R_{d_i}\,\Omega_{d_i}, \label{eq:eOmega_control}
\end{equation}
where $\Omega_{d_i} \in \R^3$ is the desired angular velocity in the body frame (set to zero in the current implementation, as no angular velocity feedforward is computed), are the primary error quantities. The attitude error $e_{R_i}$ is related to the configuration error function $\Psi_{R_i} = \frac{1}{2}\tr(I - R_{d_i}^\top R_i)$ via
\begin{equation}
  e_{R_i} = -\nabla_{R_i} \Psi_{R_i},
  \label{eq:eR_gradient}
\end{equation}
which provides a natural gradient descent direction on $\SOthree$~\cite{lee2010geometric}.

\subsubsection{Geometric Torque Law}

The full geometric control torque on $\SOthree$ is
\begin{equation}
  \tau_i = -k_R\,e_{R_i} - k_\Omega\,e_{\Omega_i} + \Omega_i \times J\Omega_i - J\bigl(\hatmap{\Omega_i}\,R_i^\top R_{d_i}\,\Omega_{d_i} - R_i^\top R_{d_i}\,\dot{\Omega}_{d_i}\bigr),
  \label{eq:full_geometric_torque}
\end{equation}
where the first two terms are the proportional-derivative control, the third is the gyroscopic compensation, and the last term is the feedforward. Under the simplification $\Omega_{d_i} \approx 0$ (valid when the desired thrust direction changes slowly relative to the attitude bandwidth), the control law reduces to
\begin{equation}
  \tau_i = -k_R\,e_{R_i} - k_\Omega\,e_{\Omega_i},
  \label{eq:simplified_torque}
\end{equation}
with $k_R = 8.0$ and $k_\Omega = 1.5$. Each component is saturated to $\abs{\tau_{i,k}} \leq \bar{\tau} = 10.0$\,N$\cdot$m. This saturation limit is conservative for the simulated platform; the observed peak torque of 1.74\,N$\cdot$m (Section~\ref{sec:results:control}) remains well within this bound. For specific hardware, $\bar{\tau}$ should be set to the platform's actuator limit.

\subsubsection{Stability Properties}

Define the Lyapunov function candidate~\cite{lee2010geometric}
\begin{equation}
  V_R = \frac{1}{2}\,k_R\,\Psi_{R_i} + \frac{1}{2}\,e_{\Omega_i}^\top J\,e_{\Omega_i} + c\,e_{R_i}^\top J\,e_{\Omega_i},
  \label{eq:lyapunov_attitude}
\end{equation}
for a sufficiently small cross-term coefficient $c > 0$. Under the simplified control law~\eqref{eq:simplified_torque}, the time derivative satisfies
\begin{equation}
  \dot{V}_R \leq -\lambda_{\min}(W)\bigl(\norm{e_{R_i}}^2 + \norm{e_{\Omega_i}}^2\bigr)
  \label{eq:Vdot_attitude}
\end{equation}
for an appropriate positive-definite matrix $W = W(k_R, k_\Omega, J)$, provided the initial attitude error satisfies $\Psi_{R_i}(0) < 2$ (i.e., the initial rotation is less than $180^\circ$ from the desired). This gives \emph{almost-global exponential} stability of $(R_i, \Omega_i) = (R_{d_i}, \Omega_{d_i})$ on the dense open set $\{\Psi_{R_i} < 2\} \subset \SOthree$~\cite{lee2010geometric}.

\subsubsection{Cascaded Stability}

The two-layer cascade (position$\to$attitude) inherits stability from the timescale separation: the attitude loop converges an order of magnitude faster than the position loop ($K_p / m_Q \approx 5$\,Hz effective bandwidth). Standard singular perturbation arguments~\cite{khalil2002nonlinear} guarantee that the composite system is locally exponentially stable when the bandwidth ratio exceeds a computable threshold, and the anti-swing control on $\Sph^2$ provides additional damping of the slow cable modes.

% ============================================================
\subsection{Layer 3: Concurrent Learning Estimator}
\label{sec:control:layer3}
% ============================================================

The payload mass $m_L$ is unknown and must be estimated online. Each quadrotor independently estimates its share $\hat{\theta}_i \approx m_L / N$ using only local measurements---cable tension, cable direction, and the estimated load acceleration. A concurrent learning adaptive law~\cite{chowdhary2010concurrent, chowdhary2013exponentially} augments the standard gradient-descent update with stored historical data, guaranteeing parameter convergence \emph{without} the persistent excitation condition required by classical adaptive control~\cite{chowdhary2015concurrent}. This is critical because cooperative hover provides insufficient excitation for classical adaptive laws.

The estimated parameter feeds into the position controller (Layer~1) as the gravity compensation term $\hat{\theta}_i(g\,e_3 + \ddot{p}_d^L)$. The full derivation of the regressor model, adaptation law, history buffer management, parameter projection, and convergence analysis is presented in Section~\ref{sec:estimation:adaptive}.

% ============================================================
\subsection{Layer 4: Extended State Observer}
\label{sec:control:layer4}
% ============================================================

The innermost estimation layer is a continuous-time third-order Extended State Observer (ESO) with bandwidth $\omega_o = 50$\,rad/s, based on the active disturbance rejection control (ADRC) framework of Han and Guo~\cite{han2009pid, guo2013active}. The ESO estimates position, velocity, and the total \emph{lumped disturbance} (including modeling errors, unmodeled dynamics, and external forces) from position measurements alone, without requiring a detailed plant model.

\subsubsection{Observer Structure}

For each spatial axis $k \in \{x, y, z\}$, the ESO models the translational dynamics as a double integrator with an unknown disturbance:
\begin{equation}
  \ddot{p}_k = b_0\,u_k + d_k(t),
  \label{eq:eso_plant}
\end{equation}
where $u_k$ is the control force component (N) along axis $k$, $b_0 = 1/m_Q = 0.667$\,kg$^{-1}$ is the nominal input gain mapping force to acceleration, and $d_k(t)$ is the lumped disturbance. The ESO is a third-order observer:
\begin{align}
  \dot{\hat{z}}_{1_k} &= \hat{z}_{2_k} + \ell_1\,(p_k - \hat{z}_{1_k}), \label{eq:eso_z1} \\
  \dot{\hat{z}}_{2_k} &= \hat{z}_{3_k} + \ell_2\,(p_k - \hat{z}_{1_k}) + b_0\,u_k, \label{eq:eso_z2} \\
  \dot{\hat{z}}_{3_k} &= \ell_3\,(p_k - \hat{z}_{1_k}), \label{eq:eso_z3}
\end{align}
where $\hat{z}_{1_k}$, $\hat{z}_{2_k}$, and $\hat{z}_{3_k}$ estimate position, velocity, and disturbance, respectively, and $\ell_1, \ell_2, \ell_3$ are the observer gains.

\subsubsection{Gain Design: Bandwidth Parameterization}

The gains are chosen to place all three observer poles at $-\omega_o$, yielding
\begin{equation}
  \ell_1 = 3\omega_o, \qquad \ell_2 = 3\omega_o^2, \qquad \ell_3 = \omega_o^3,
  \label{eq:eso_gains}
\end{equation}
where $\omega_o = 50$\,rad/s is the observer bandwidth. This \emph{bandwidth parameterization}~\cite{guo2013active} reduces the tuning to a single scalar: increasing $\omega_o$ improves disturbance tracking but amplifies measurement noise, while decreasing $\omega_o$ provides better noise rejection at the cost of slower disturbance estimation. The characteristic polynomial of the observer error dynamics is $(\lambda + \omega_o)^3 = 0$, giving a uniform settling time of $t_s \approx 5/\omega_o = 0.1$\,s.

\subsubsection{Disturbance Estimate}

The disturbance estimate $\hat{d}_i = [\hat{z}_{3_x},\, \hat{z}_{3_y},\, \hat{z}_{3_z}]^\top$ is saturated component-wise to $\abs{\hat{d}_{i,k}} \leq \bar{d} = 20$\,m/s$^2$ to prevent unbounded estimates during transients. This estimate is fed forward via~\eqref{eq:eso_ff}, providing feedforward disturbance compensation with estimation lag bounded by $5/\omega_o = 0.1$\,s.

\subsubsection{Convergence Properties}

Under the assumption that $d_k(t)$ is differentiable with bounded derivative, the estimation errors $\tilde{z}_{j_k} = z_{j_k} - \hat{z}_{j_k}$ satisfy~\cite{guo2013active}
\begin{equation}
  \abs{\tilde{z}_{j_k}(t)} \leq c_j\,\omega_o^{j-4}\,\sup_{\tau \leq t}\abs{\dot{d}_k(\tau)} + \mathcal{O}(e^{-\omega_o t}), \quad j = 1,2,3,
  \label{eq:eso_convergence}
\end{equation}
for constants $c_j > 0$. In particular, the disturbance estimation error scales as $\mathcal{O}(\sup\abs{\dot{d}}/\omega_o)$, showing that faster observer bandwidth yields tighter estimation.

% ============================================================
\subsection{CBF Safety Filter}
\label{sec:control:cbf}
% ============================================================

A Control Barrier Function (CBF) safety filter~\cite{ames2017control, ames2019control} is applied as a minimally invasive post-processing step to enforce operational constraints while staying as close to the nominal GPAC control signal as possible.

\subsubsection{Barrier Function Design}

We define five barrier functions encoding the physical safety constraints. For each constraint $h : \R^n \to \R$, the safe set is $\mathcal{C} = \{x \mid h(x) \geq 0\}$:

\paragraph{Cable Tautness} The most critical constraint ensures that cable tension remains within bounds:
\begin{equation}
  h_T^{\text{low}} = T_i - T_{\min}, \qquad h_T^{\text{up}} = T_{\max} - T_i,
  \label{eq:cbf_tautness}
\end{equation}
with $T_{\min} = 2.0$\,N and $T_{\max} = 60.0$\,N.

\paragraph{Cable Angle} The cable deviation from vertical is bounded:
\begin{equation}
  h_\theta = \cos\theta_{\max} - \cos\theta_i, \qquad \theta_i = \arccos(q_i^\top (-e_3)),
  \label{eq:cbf_angle}
\end{equation}
with $\theta_{\max} = 0.6$\,rad ($\approx 34^\circ$).

\paragraph{Swing Rate} The cable angular velocity is bounded:
\begin{equation}
  h_\omega = \omega_{\max}^2 - \norm{\omega_{q_i}}^2,
  \label{eq:cbf_swing}
\end{equation}
with $\omega_{\max} = 1.5$\,rad/s.

\paragraph{Quadrotor Tilt} The tilt angle is constrained:
\begin{equation}
  h_{\text{tilt}} = \cos\phi_{\max} - \cos\phi_i, \qquad \phi_i = \arccos(e_3^\top R_i\,e_3),
  \label{eq:cbf_tilt}
\end{equation}
with $\phi_{\max} = 0.5$\,rad ($\approx 29^\circ$).

\paragraph{Inter-Agent Separation} Collision avoidance between quadrotors:
\begin{equation}
  h_{\mathrm{col},ij} = \norm{p_i - p_j}^2 - d_{\min}^2,
  \label{eq:cbf_collision}
\end{equation}
with $d_{\min} = 0.8$\,m.

\begin{remark}[Physical rationale for safety constraints]
Each barrier function corresponds to a specific operational hazard:
cable tautness prevents uncontrolled payload swing upon re-engagement after slack;
cable angle limits prevent entanglement and maintain a controllable suspension geometry;
swing rate bounds prevent resonant excitation of the bead-chain cable modes;
tilt limits ensure sufficient vertical thrust margin for altitude maintenance;
and collision avoidance prevents catastrophic inter-agent contact.
\end{remark}

\begin{table}[t]
  \centering
  \caption{CBF Safety Filter Parameters}
  \label{tab:cbf_params}
  \begin{tabular}{lcc}
    \hline
    \textbf{Constraint} & \textbf{$\alpha_j$} & \textbf{Threshold} \\
    \hline
    Cable tautness (lower) & 3.0 & $T_{\min} = 2.0$\,N \\
    Cable tautness (upper) & 3.0 & $T_{\max} = 60.0$\,N \\
    Cable angle & 2.0 & $\theta_{\max} = 0.6$\,rad \\
    Swing rate & 1.5 & $\omega_{\max} = 1.5$\,rad/s \\
    Quadrotor tilt & 4.0 & $\phi_{\max} = 0.5$\,rad \\
    Collision avoidance & 3.0 & $d_{\min} = 0.8$\,m \\
    \hline
  \end{tabular}
\end{table}

\subsubsection{Safety QP and Formal Analysis}

The safety filter solves a quadratic program at each control cycle that minimally modifies the nominal thrust to maintain all barrier constraints. The full QP formulation, Lie derivative derivations for each transport-specific constraint, Higher-Order CBF treatment of relative-degree-two constraints (cable angle, tilt, collision avoidance), the Butterworth-filtered tension rate estimation, the input-to-state safety guarantee incorporating the ESO disturbance estimate, and the compatibility proof with the geometric attitude controller are presented in Section~\ref{sec:safety}.