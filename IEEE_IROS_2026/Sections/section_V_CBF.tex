% !TEX root = ../Main.tex
A modular CBF layer enforces operational constraints as a minimally invasive post-processing step, modifying the nominal GPAC output only when necessary. Each barrier function corresponds directly to a cooperative transport hazard identified in Section~\ref{sec:intro}, providing runtime supervision within verified safety envelopes.

\subsection{Barrier Function Design}

The safety filter operates on the force vector $f_i \in \R^3$ from Layer~1. Abstracting the translational dynamics~\eqref{eq:quad_trans} as $m_Q\ddot{p}_i = f_i + w_i(t)$ (affine in the control input, with lumped disturbance $w_i$ estimated by the ESO), five barrier functions encode the safe set $\mathcal{C}_k = \{x \mid h_k(x) \geq 0\}$:
\begin{align}
  h_T^{\text{low}} &= T_i - T_{\min},\;\; h_T^{\text{up}} = T_{\max} - T_i & \text{(tautness)}, \label{eq:h_taut}\\
  h_\theta &= \cos\theta_{\max} - \cos\theta_i & \text{(cable angle)}, \label{eq:h_angle}\\
  h_\omega &= \omega_{\max}^2 - \norm{\omega_{q_i}}^2 & \text{(swing rate)}, \label{eq:h_swing}\\
  h_{\text{tilt}} &= \cos\phi_{\max} - \cos\phi_i & \text{(vehicle tilt)}, \label{eq:h_tilt}\\
  h_{\text{col}} &= \norm{p_i - p_j}^2 - d_{\min}^2 & \text{(collision)}, \label{eq:h_col}
\end{align}
with $T_{\min}\!=\!2$\,N, $T_{\max}\!=\!60$\,N, $\theta_{\max}\!=\!0.6$\,rad ($34.4^\circ$), $\omega_{\max}\!=\!1.5$\,rad/s, $\phi_{\max}\!=\!0.5$\,rad ($28.6^\circ$), $d_{\min}\!=\!0.8$\,m. For the collision barrier~\eqref{eq:h_col}, the second derivative involves both $f_i$ and $f_j$; in the decentralized setting, agent~$i$ treats the neighbor's acceleration as a bounded disturbance and enforces a conservative one-sided HOCBF constraint using only its own force input, with the disturbance absorbed into the ISSf margin.

The tautness barriers~\eqref{eq:h_taut} are relative-degree-one. Cable angle, tilt, and collision are relative-degree-two and use HOCBFs~\cite{xiao2022control, ames2019control}: defining $\psi_1 = \dot{h} + \alpha_1 h$, the constraint $\dot{\psi}_1 + \alpha_2\psi_1 \geq 0$ becomes relative-degree-one in the control, with the safe set $\{\psi_1 \geq 0\} \cap \{h \geq 0\}$ rendered forward-invariant.

\subsection{Safety QP with ISSf Guarantees}

The safety filter solves, at each control cycle:
\begin{equation}
  f_{\text{safe}} = \argmin_{f,\,\delta}\;\norm{f - f_{\text{nom}}}^2 + \lambda\!\sum_j\!\delta_j^2 \;\;\text{s.t.}\;\; \dot{h}_j + \alpha_j h_j \geq -\mu_j - \delta_j,
  \label{eq:cbf_qp}
\end{equation}
with $\lambda = 100$ penalizing relaxation. In the presence of disturbances, hard forward invariance of $\{h \geq 0\}$ is unrealistic. We employ the Input-to-State Safety (ISSf) framework, coupling the robustness margin to the ESO estimate:
\begin{equation}
  \dot{h}_j + \alpha_j h_j \geq -\mu_{\text{base}} - \kappa_d\norm{\hat{d}_i},
  \label{eq:issf}
\end{equation}
with $\mu_{\text{base}} = 2.0$, $\kappa_d = 1.5$. When the ESO reports large disturbances, the margin grows and the filter activates earlier. The steady-state violation bound is:
\begin{equation}
  h_j(t) \geq -(\mu_{\text{base}} + \kappa_d\bar{d})/\alpha_j.
  \label{eq:issf_bound}
\end{equation}

\begin{figure}[t]
  \centering
  \includegraphics[width=\columnwidth]{Figures/fig_cbf_activation.png}
  \caption{CBF activation timeline. \textit{Top:} Tension margin (distance to $T_{\min}$) for all cables. \textit{Middle:} Cable angle margin (distance to $\theta_{\max} = 34.4^\circ$); shaded regions indicate CBF activation during aggressive cornering (1.7\% of time). \textit{Bottom:} Tilt margin (never violated). Orange regions denote constraint activation.}
  \label{fig:cbf_activation}
\end{figure}

The tautness constraint requires $\dot{T}_i$, estimated via a second-order Butterworth low-pass ($f_c = 15$\,Hz, $-40$\,dB/dec) applied to finite-difference tension rate, attenuating cable vibrations (${\sim}55$\,Hz) by ${\sim}22$\,dB with ${\sim}11$\,ms group delay.

The implementation uses sequential gradient projection with priority ordering (tautness $>$ angle $>$ tilt $>$ swing $>$ collision) rather than a full QP solver. This approximation is equivalent to~\eqref{eq:cbf_qp} when at most one constraint is active---the typical case, as the CBF is active only 1.7--3.2\% of total simulation time. Consequently, the ISSf guarantees of~\eqref{eq:issf_bound} hold during single-constraint activation. When multiple constraints activate simultaneously (rare; concentrated in $<$0.3\% of time during the sharpest cornering), the sequential method enforces the explicit priority ordering above, producing a feasible but potentially suboptimal solution; the slack variables $\delta_j$ ensure that lower-priority constraints degrade gracefully. The per-agent cost is $O(N_c)$ with $N_c = 6$ constraints (${\sim}1200$\,FLOPs/cycle).

\subsection{Compatibility with Geometric Attitude Control}

The safety filter modifies the force direction, which changes the desired rotation $R_d$. A critical requirement is that this perturbation does not exit the almost-global stability region of the geometric controller.

\begin{theorem}[Safety-Stability Compatibility]\label{thm:compatibility}
Suppose the tilt barrier~\eqref{eq:h_tilt} with $\phi_{\max}=0.5$\,rad is enforced. Then the attitude error between the nominal and safe desired rotations satisfies:
\begin{equation}
  \Psi_R(R_d^{\text{nom}}, R_d^{\text{safe}}) \leq 1 - \cos(2\phi_{\max}) \approx 0.46 < 1,
  \label{eq:compat}
\end{equation}
and the geometric attitude controller retains almost-global exponential stability.
\end{theorem}

\begin{figure}[t]
  \centering
  \includegraphics[width=\columnwidth]{Figures/fig_safety_constraints.png}
  \caption{Safety constraint satisfaction. \textit{Top:} Cable angle from vertical for each cable, with the CBF limit $\theta_{\max} = 34.4^\circ$ (dashed). Brief excursions above the limit during aggressive cornering remain within the ISSf margin. \textit{Bottom:} Quadrotor tilt angle with limit $\phi_{\max} = 28.6^\circ$ (dashed); the constraint is never violated.}
  \label{fig:safety_constraints}
\end{figure}

\begin{proof}
The tilt constraint bounds $\phi_i \leq \phi_{\max}$, so the angle between $f_{\text{nom}}$ and $f_{\text{safe}}$ satisfies $\vartheta \leq 2\phi_{\max} = 1.0$\,rad. Since $R_d$ aligns $b_{3c} = f/\norm{f}$~\eqref{eq:Rd_extract}, a rotation by $\vartheta$ gives $\Psi_R(R_d^{\text{nom}}, R_d^{\text{safe}}) = 1 - \cos\vartheta \leq 0.46$. As $\Psi_R(R_i, R_d^{\text{nom}}) \to 0$ exponentially (Proposition~\ref{prop:lyap}), subadditivity of $\Psi_R$~\cite{lee2010geometric} yields $\Psi_R(R_i, R_d^{\text{safe}}) < \varepsilon + 0.46 < 2$ after settling (${\sim}0.1$\,s).
\end{proof}

\begin{remark}[Timescale separation]\label{rem:timescale}
The compatibility result relies on three well-separated timescales: the \emph{fast} attitude loop ($k_R/J \approx 200$\,rad/s, settling ${\sim}5$\,ms), the \emph{medium} safety filter (Butterworth cutoff $2\pi \times 15 \approx 94$\,rad/s), and the \emph{slow} position/cable dynamics ($\sqrt{g/L} \approx 3$\,rad/s). Since the fast dynamics settle (${\sim}5$\,ms) an order of magnitude faster than force modifications evolve (${\sim}50$--$200$\,ms), standard singular perturbation analysis~\cite{khalil2002nonlinear} ensures the attitude controller tracks the slowly varying $R_d^{\text{safe}}$ without losing stability. This bandwidth hierarchy $\omega_{\text{att}} \gg \omega_{\text{CBF}} \gg \omega_{\text{pos}}$ is the structural condition making the layered architecture composable.
\end{remark}
