% !TEX root = ../Main.tex
GPAC is validated in Drake-based~\cite{drake2024} multibody simulation (parameters in Table~\ref{tab:params}). Physics runs at 5000\,Hz with semi-implicit Euler, resolving cable vibration ($\sim$55\,Hz) with $90\times$ oversampling. Cable rest lengths are asymmetric ([0.914, 1.105, 0.995]\,m, up to 19\% variation). Multi-rate timing matches the GPAC layer structure. Dryden wind and the full sensor suite complete the environment. All loops close through the ESKF, not ground truth, ensuring sensor-in-the-loop realism.

The 50\,s benchmark exercises all hazards: pre-lift hover with slack-to-taut transitions, ascent with impulsive loading, figure-eight maneuvering with aggressive cornering that excites pendular swing, and controlled descent. The figure-eight spans a $6.0 \times 3.0$\,m workspace at payload altitudes of 1.6--2.3\,m, with path length exceeding 20\,m. Minimum-jerk interpolation enforces $\norm{v_d}\!\leq\!1$\,m/s, $\norm{a_d}\!\leq\!2$\,m/s$^2$.

\begin{figure}[t]
  \centering
  \includegraphics[width=0.84\columnwidth]{Figures/fig_trajectory_3d.png}
  \caption{3D trajectory. The max velocity of the load is 1\,m/s. Gray dashed lines: cable attachments. Asymmetric formation accommodates 19\% uncertainty in cable length.}
  \label{fig:trajectory_3d}
\end{figure}

\subsection{Tracking Performance}

Table~\ref{tab:tracking} summarizes payload tracking errors for the baseline seed. The overall 3D RMSE of 22.9\,cm corresponds to 3.3\% of the workspace diagonal---well within one cable length. This is achieved without centralized coordination, under wind disturbance and 19\% cable asymmetry. A 15-seed Monte Carlo confirms representativeness (6.3\% CV). Horizontal error dominates because pendulum dynamics amplify lateral disturbances, while vertical error is well-controlled via direct thrust--altitude coupling and barometer-aided estimation. Peak errors occur during figure-eight cornering; performance improves as concurrent learning accumulates data.

\begin{table}[t]
  \centering
  \caption{Payload tracking errors (cm).}
  \label{tab:tracking}
  \begin{tabular}{@{}lcccccc@{}}
    \toprule
    & \multicolumn{2}{c}{\textbf{Horiz.}} & \multicolumn{2}{c}{\textbf{Vert.}} & \multicolumn{2}{c}{\textbf{3D}} \\
    \textbf{Phase} & RMSE & Max & RMSE & Max & RMSE & Max \\
    \midrule
    Ascent (2--6\,s) & 8.9 & 26.7 & 11.0 & 25.9 & 14.2 & 28.3 \\
    Fig-8 right (7--20\,s) & 26.6 & 79.9 & 4.0 & 10.7 & 26.9 & 80.3 \\
    Fig-8 left (24--36\,s) & 17.6 & 37.5 & 3.6 & 9.4 & 18.0 & 37.7 \\
    Descent (39--43\,s) & 13.1 & 27.1 & 9.0 & 16.4 & 16.0 & 27.3 \\
    Post-descent (43--50\,s) & 7.7 & 12.4 & 1.6 & 4.8 & 7.9 & 12.5 \\
    \midrule
    \textbf{Overall} & 22.2 & 79.9 & 5.4 & 25.9 & \textbf{22.9} & 80.3 \\
    \bottomrule
  \end{tabular}
\end{table}

\begin{figure}[t]
  \centering
  \includegraphics[width=0.84\columnwidth]{Figures/fig_tracking_error.png}
  \caption{Payload tracking error. \textit{Top:} 3D error (RMSE 22.9\,cm); peaks (80.3\,cm) during aggressive cornering. \textit{Bottom:} Components. The right loop (7--20\,s) shows larger errors before CL converges.}
  \label{fig:tracking_error}
\end{figure}

\subsection{Estimation and Adaptive Convergence}

The ESKF achieves quadrotor position RMSE consistent with the GPS noise floor, with increased error during cornering. Low variance across Monte Carlo seeds confirms that ESKF accuracy is determined by sensor noise, not cable geometry. The decentralized load estimator (49.5\,cm RMSE) is $4\times$ worse than a centralized oracle (12.4\,cm), reflecting the single-cable observability limitation discussed in Section~\ref{sec:estimation}. The centralized baseline requires $\sim$29\,kbps bandwidth.

Concurrent learning converges within $\sim$8\,s, accumulating $\sim$30 informative samples during ascent. Without concurrent learning ($\rho = 0$), settling extends significantly with larger oscillation (see Table~\ref{tab:failure_modes}).

\begin{figure}[t]
  \centering
  \includegraphics[width=0.84\columnwidth]{Figures/fig_estimation_error.png}
  \caption{Estimation performance. \textit{Top:} ESKF error (7.1\,cm RMSE, max 22.3\,cm during cornering). \textit{Middle:} Decentralized load estimate (49.5\,cm RMSE), limited by single-cable observability. \textit{Bottom:} Per-quadrotor mass estimates converging to $m_L/N$.}
  \label{fig:estimation_error}
\end{figure}

Table~\ref{tab:failure_modes} consolidates the hazard-to-mitigation mapping with constraint enforcement data. Cable angle CBF activates most frequently; the observed excursion ($48.1^\circ$ vs.\ $34.4^\circ$ limit) and swing rate ($1.6$ vs.\ $1.5$\,rad/s) are consistent with the ISSf bound, which predicts margin violations of this magnitude under peak disturbance. Tilt and collision constraints are never violated ($\geq 1.04$\,m clearance, 30\% above $d_{\min}$). The CBF activates infrequently and increases tracking RMSE by only 2\%.

Table~\ref{tab:ablation} isolates each component's contribution. The ESO provides the largest benefit (+43\% degradation when removed), followed by concurrent learning (+33\%). This ranking indicates that wind disturbance rejection is the dominant contributor to tracking accuracy in this scenario, consistent with the Dryden turbulence intensities used. The CBF imposes minimal tracking cost while substantially reducing peak cable angles. The centralized estimation baseline reduces RMSE by 22\% but requires inter-agent communication.

\begin{table}[t]
  \centering
  \caption{Ablation results: payload tracking RMSE (cm).}
  \label{tab:ablation}
  \begin{tabular}{@{}lccc@{}}
    \toprule
    \textbf{Configuration} & \textbf{RMSE} & $\boldsymbol{\Delta}$ & \textbf{Max Angle} \\
    \midrule
    Full GPAC (baseline) & 22.9 & --- & 48.1$^\circ$ \\
    No concurrent learning & 31.7 & +33\% & 42.1$^\circ$ \\
    No ESO feedforward & 34.1 & +43\% & 44.8$^\circ$ \\
    No CBF safety filter & 24.2 & +2\% & 52.0$^\circ$ \\
    Centralized estimation & 18.5 & $-$22\% & 38.2$^\circ$ \\
    \bottomrule
  \end{tabular}
\end{table}

Per-agent computation totals $\sim$0.7\,MFLOP/s, dominated by the ESKF ($\sim$2000\,FLOPs/cycle at 200\,Hz) with the CBF adding $\sim$1200\,FLOPs/cycle. This fits comfortably on ARM Cortex-M7 class processors (400\,MHz, single-precision FPU) with margin for communication and sensing.

\begin{table}[t]
  \centering
  \caption{Cable tension statistics (N) during steady-state flight. Asymmetric cable lengths produce unequal load sharing.}
  \label{tab:tension_stats}
  \begin{tabular}{@{}lccccc@{}}
    \toprule
    \textbf{Cable} & $\boldsymbol{L_i}$ \textbf{(m)} & \textbf{Mean} & \textbf{Std} & \textbf{Min} & \textbf{Max} \\
    \midrule
    0 & 0.914 & 14.88 & 3.31 & 1.52 & 26.34 \\
    1 & 1.105 & 10.23 & 3.14 & 2.00 & 22.16 \\
    2 & 0.995 & 13.03 & 2.75 & 6.07 & 19.96 \\
    \bottomrule
  \end{tabular}
\end{table}

\begin{figure}[t]
  \centering
  \includegraphics[width=0.84\columnwidth]{Figures/fig_cable_tensions.png}
  \caption{Cable tensions. Asymmetric lengths yield unequal load sharing accommodated without coordination.}
  \label{fig:cable_tensions}
  \vspace{-5pt}
\end{figure}

\begin{table}[t]
  \centering
  \caption{Hazard-to-mitigation mapping with constraint limits and measured performance.}
  \label{tab:failure_modes}
  \begin{tabular}{@{}llccc@{}}
    \toprule
    \textbf{Hazard} & \textbf{Mitigation} & \textbf{Limit} & \textbf{Observed}\\
    \midrule
    Mass unknown & CL estimation & --- & $\hat{\theta} \!\to\! m_L/N$, 8\,s\\
    Wind disturbance & ESO feedfwd & --- & +43\% w/o\\
    Cable slack & CBF tension & {[2, 60]\,N} & {[1.5, 26.3]\,N}\\
    Cable angle & CBF + anti-swing & $34.4^\circ$ & $48.1^\circ$\\
    Excessive tilt & CBF tilt & $28.6^\circ$ & $25.1^\circ$\\
    Swing rate & CBF swing & 1.5\,rad/s & 1.6\,rad/s\\
    Collision & CBF separation & 0.8\,m & $\geq 1.04$\,m\\
    Sensor noise & ESKF fusion & --- & 6.9\,cm RMSE\\
    \bottomrule
  \end{tabular}
\end{table}
