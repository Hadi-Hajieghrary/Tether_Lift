% !TEX root = ../Main.tex
We validated GPAC using a Drake-based~\cite{drake2024} multibody simulation, with parameters listed in Table~\ref{tab:params}. The physics engine runs at 5000 Hz using a semi-implicit Euler method, and cable vibrations (around 55 Hz) are resolved with 90 times oversampling. The cable rest lengths are asymmetric ([0.914, 1.105, 0.995] m), showing up to 19\% variation. The multi-rate timing is set to match the GPAC layer structure. The simulation environment includes Dryden wind and a full sensor suite. All control loops are closed using the ESKF rather than ground truth, ensuring sensor-in-the-loop realism.

The overall 3D RMSE is 22.9 cm, which is 3.3\% of the workspace diagonal and well within one cable length. This result is achieved without centralized coordination, even with wind disturbance and 19\% cable asymmetry. A 15-seed Monte Carlo test confirms the results are representative (6.3\% CV). Most of the error is horizontal because pendulum dynamics amplify lateral disturbances, while vertical error is well controlled through direct thrust-altitude coupling and barometer-aided estimation. The largest errors occur during figure-eight cornering, but performance improves as concurrent learning collects more data.

\begin{figure}[t]
  \centering
  \includegraphics[width=0.84\columnwidth]{Figures/fig_trajectory_3d.png}
  \caption{3D trajectory of the load and the quadcopters. The max velocity of the load is 1\,m/s.}
  \label{fig:trajectory_3d}
\end{figure}

\begin{table}[t]
  \centering
  \caption{Payload tracking errors (cm).}
  \label{tab:tracking}
  \begin{tabular}{@{}lcccccc@{}}
    \toprule
    & \multicolumn{2}{c}{\textbf{Horiz.}} & \multicolumn{2}{c}{\textbf{Vert.}} & \multicolumn{2}{c}{\textbf{3D}} \\
    \textbf{Phase} & RMSE & Max & RMSE & Max & RMSE & Max \\
    \midrule
    Ascent (2--6\,s) & 8.9 & 26.7 & 11.0 & 25.9 & 14.2 & 28.3 \\
    Fig-8 right (7--20\,s) & 26.6 & 79.9 & 4.0 & 10.7 & 26.9 & 80.3 \\
    Fig-8 left (24--36\,s) & 17.6 & 37.5 & 3.6 & 9.4 & 18.0 & 37.7 \\
    Descent (39--43\,s) & 13.1 & 27.1 & 9.0 & 16.4 & 16.0 & 27.3 \\
    Post-descent (43--50\,s) & 7.7 & 12.4 & 1.6 & 4.8 & 7.9 & 12.5 \\
    \midrule
    \textbf{Overall} & 22.2 & 79.9 & 5.4 & 25.9 & \textbf{22.9} & 80.3 \\
    \bottomrule
  \end{tabular}
\end{table}

\begin{figure}[t]
  \centering
  \includegraphics[width=0.84\columnwidth]{Figures/fig_tracking_error.png}
  \caption{Payload tracking error. \textit{Top:} 3D error (RMSE 22.9\,cm); peaks (80.3\,cm) during aggressive cornering. \textit{Bottom:} Components. The right loop (7--20\,s) shows larger errors before CL converges.}
  \label{fig:tracking_error}
  \vspace{-5pt}
\end{figure}

\begin{figure}[t]
  \centering
  \includegraphics[width=0.84\columnwidth]{Figures/fig_estimation_error.png}
  \caption{Estimation performance. \textit{Top:} ESKF error (7.1\,cm RMSE, max 22.3\,cm during cornering). \textit{Middle:} Decentralized load estimate (49.5\,cm RMSE), limited by single-cable observability. \textit{Bottom:} Per-quadrotor mass estimates converging to $m_L/N$.}
  \label{fig:estimation_error}
  \vspace{-5pt}
\end{figure}

Table~\ref{tab:tracking} shows payload tracking errors for the baseline seed. The overall 3D RMSE is 22.9 cm, which is 3.3\% of the workspace diagonal and well within one cable length. This result is achieved without centralized coordination, even with wind disturbance and 19\% cable asymmetry. A 15-seed Monte Carlo test confirms the results are representative (6.3\% CV). Most of the error is horizontal because pendulum dynamics amplify lateral disturbances, while vertical error remains low thanks to direct thrust-altitude control and barometer-aided estimation. The largest errors occur during figure-eight cornering, but performance improves as concurrent learning collects more data.

The ESKF gives quadrotor position RMSE that matches the GPS noise floor, though errors increase during cornering. Low variance across Monte Carlo seeds shows that ESKF accuracy depends on sensor noise, not cable geometry. The decentralized load estimator has a 49.5 cm RMSE, which is 4 times worse than the centralized oracle's 12.4 cm RMSE. This difference is due to the single-cable observability limit discussed in Section~\ref{sec:estimation}. The centralized baseline needs about 29 kbps of bandwidth.

Concurrent learning converges in about 8 seconds, collecting around 30 useful samples during ascent. Without concurrent learning ($\rho = 0$), settling takes much longer and oscillations are larger (see Table~\ref{tab:failure_modes}).

Table~\ref{tab:failure_modes} shows the hazard-to-mitigation mapping and constraint enforcement data. The cable angle CBF is activated most often; the observed excursion (48.1° compared to the 34.4° limit) and swing rate (1.6 vs. 1.5 rad/s) match the ISSf bound, which predicts margin violations of this size during peak disturbance. Tilt and collision constraints are never violated (clearance is at least 1.04 m, which is 30\% above $d_{\min}$). The CBF is rarely activated and only increases tracking RMSE by 2\%.

Table~\ref{tab:ablation} breaks down each component's impact. The ESO provides the greatest benefit, as removing it results in a 43\% drop in performance. Concurrent learning is next, with a 33\% decrease. This order shows that handling wind disturbance is the main factor affecting tracking accuracy here, which aligns with the Dryden turbulence intensities used. The CBF has little effect on tracking cost but greatly reduces peak cable angles. The centralized estimation baseline lowers RMSE by 22\%, but it requires agents to communicate.

\begin{table}[t]
  \centering
  \caption{Ablation results: payload tracking RMSE (cm).}
  \label{tab:ablation}
  \begin{tabular}{@{}lccc@{}}
    \toprule
    \textbf{Configuration} & \textbf{RMSE} & $\boldsymbol{\Delta}$ & \textbf{Max Angle} \\
    \midrule
    Full GPAC (baseline) & 22.9 & --- & 48.1$^\circ$ \\
    No concurrent learning & 31.7 & +33\% & 42.1$^\circ$ \\
    No ESO feedforward & 34.1 & +43\% & 44.8$^\circ$ \\
    No CBF safety filter & 24.2 & +2\% & 52.0$^\circ$ \\
    Centralized estimation & 18.5 & $-$22\% & 38.2$^\circ$ \\
    \bottomrule
  \end{tabular}
\end{table}

\begin{table}[t]
  \centering
  \caption{Cable tension statistics (N) during steady-state flight. Asymmetric cable lengths produce unequal load sharing.}
  \label{tab:tension_stats}
  \begin{tabular}{@{}lccccc@{}}
    \toprule
    \textbf{Cable} & $\boldsymbol{L_i}$ \textbf{(m)} & \textbf{Mean} & \textbf{Std} & \textbf{Min} & \textbf{Max} \\
    \midrule
    0 & 0.914 & 14.88 & 3.31 & 1.52 & 26.34 \\
    1 & 1.105 & 10.23 & 3.14 & 2.00 & 22.16 \\
    2 & 0.995 & 13.03 & 2.75 & 6.07 & 19.96 \\
    \bottomrule
  \end{tabular}
\end{table}

\begin{figure}[t]
  \centering
  \includegraphics[width=0.84\columnwidth]{Figures/fig_cable_tensions.png}
  \caption{Cable tensions. Asymmetric lengths yield unequal load sharing accommodated without coordination.}
  \label{fig:cable_tensions}
  \vspace{-5pt}
\end{figure}

\begin{table}[t]
  \centering
  \caption{Hazard-to-mitigation mapping with constraint limits and measured performance.}
  \label{tab:failure_modes}
  \begin{tabular}{@{}llccc@{}}
    \toprule
    \textbf{Hazard} & \textbf{Mitigation} & \textbf{Limit} & \textbf{Observed}\\
    \midrule
    Mass unknown & CL estimation & --- & $\hat{\theta} \!\to\! m_L/N$, 8\,s\\
    Wind disturbance & ESO feedfwd & --- & +43\% w/o\\
    Cable slack & CBF tension & {[2, 60]\,N} & {[1.5, 26.3]\,N}\\
    Cable angle & CBF + anti-swing & $34.4^\circ$ & $48.1^\circ$\\
    Excessive tilt & CBF tilt & $28.6^\circ$ & $25.1^\circ$\\
    Swing rate & CBF swing & 1.5\,rad/s & 1.6\,rad/s\\
    Collision & CBF separation & 0.8\,m & $\geq 1.04$\,m\\
    Sensor noise & ESKF fusion & --- & 6.9\,cm RMSE\\
    \bottomrule
  \end{tabular}
\end{table}
