Section~\ref{sec:control:cbf} defined the five barrier functions and the CBF parameter table that encode the operational safety envelope for cable-suspended transport. This section provides the formal analysis: we state the control barrier function (CBF) conditions, derive the Lie derivatives for each constraint specific to the cable-suspended transport problem, handle the relative-degree mismatch via Higher-Order CBFs, introduce the Butterworth-filtered tension rate estimation that enables smooth constraint enforcement, establish an Input-to-State Safety (ISSf) guarantee incorporating the ESO disturbance estimate, and prove compatibility with the geometric attitude controller.

% ============================================================
\subsection{Preliminaries: Control Barrier Functions}
\label{sec:safety:prelim}
% ============================================================

We briefly recall the CBF framework following~\cite{ames2017control, ames2019control}. Consider a control-affine system
\begin{equation}
  \dot{x} = f(x) + g(x)\,u, \qquad x \in \R^n,\; u \in \mathcal{U} \subset \R^m,
  \label{eq:affine_system}
\end{equation}
and a safe set $\mathcal{C} = \{x \in \R^n \mid h(x) \geq 0\}$ defined by a continuously differentiable function $h : \R^n \to \R$.

\begin{definition}[Control Barrier Function~\cite{ames2019control}]
\label{def:cbf}
A continuously differentiable function $h : \R^n \to \R$ is a \emph{control barrier function} for the system~\eqref{eq:affine_system} on the set $\mathcal{C}$ if there exists a class-$\mathcal{K}_\infty$ function $\alpha$ such that
\begin{equation}
  \sup_{u \in \mathcal{U}} \bigl[L_f h(x) + L_g h(x)\,u\bigr] \geq -\alpha\bigl(h(x)\bigr), \qquad \forall\, x \in \mathcal{C},
  \label{eq:cbf_condition}
\end{equation}
where $L_f h = \nabla h \cdot f$ and $L_g h = \nabla h \cdot g$ are the Lie derivatives.
\end{definition}

The linear choice $\alpha(r) = \alpha_0\,r$ with $\alpha_0 > 0$ yields the constraint $\dot{h}(x, u) + \alpha_0\,h(x) \geq 0$, which renders $\mathcal{C}$ forward-invariant: if $x(0) \in \mathcal{C}$, then $x(t) \in \mathcal{C}$ for all $t \geq 0$ under any Lipschitz controller satisfying~\eqref{eq:cbf_condition}. The parameter $\alpha_0$ governs the rate at which the state is driven away from the boundary $\partial\mathcal{C}$, with larger values providing more aggressive constraint enforcement.

% ============================================================
\subsection{System Abstraction for the Safety Filter}
\label{sec:safety:abstraction}
% ============================================================

The CBF safety filter operates on the force vector $f_i \in \R^3$ produced by Layer~1 of the GPAC controller. To cast the multi-body dynamics into the affine form~\eqref{eq:affine_system}, we abstract each quadrotor's translational dynamics~\eqref{eq:quad_translational} as
\begin{equation}
  m_Q\,\ddot{p}_i = f_i + w_i(t),
  \label{eq:abstracted_dynamics}
\end{equation}
where $f_i = f_i\,R_i\,e_3$ is the realized thrust in world frame and $w_i(t)$ collects all unmodeled forces (gravity, cable tension, wind). Defining the state $\xi_i = [p_i^\top,\, v_i^\top]^\top \in \R^6$, the dynamics become
\begin{equation}
  \dot{\xi}_i = \underbrace{\begin{bmatrix} v_i \\ w_i / m_Q \end{bmatrix}}_{f_0(\xi_i, t)} + \underbrace{\begin{bmatrix} 0 \\ I_3 / m_Q \end{bmatrix}}_{g_0}\,f_i,
  \label{eq:abstracted_affine}
\end{equation}
which is affine in the control input $f_i$. The unmatched disturbance $w_i(t)$ is estimated by the ESO (Section~\ref{sec:control:layer4}) and incorporated into the robustness margin.

% ============================================================
\subsection{Lie Derivative Computation for Transport-Specific Barriers}
\label{sec:safety:lie}
% ============================================================

We derive the Lie derivatives for each barrier function defined in Section~\ref{sec:control:cbf}, identifying the relative degree and computing the gradients with respect to the force input.

\subsubsection{Tautness Barriers (Relative Degree 1)}

For $h_T^{\mathrm{low}} = T_i - T_{\min}$, the time derivative is
\begin{equation}
  \dot{h}_T^{\mathrm{low}} = \dot{T}_i = \frac{\partial T_i}{\partial p_i}\,\dot{p}_i + \frac{\partial T_i}{\partial \dot{p}_i}\,\ddot{p}_i.
  \label{eq:tautness_lie_raw}
\end{equation}
From the spring-damper cable model~\eqref{eq:tension_law}, the tension depends on the segment stretch $\Delta_1 = \ell_1 - L_0$ and stretch rate $\dot{\Delta}_1$. On timescales shorter than the spring period ($\omega_k = \sqrt{k_s/m_b} \gg \omega_o$), the damping-dominated response gives an effective gradient:
\begin{equation}
  \nabla_{f_i} \dot{T}_i \approx \frac{c_s}{m_Q}\bigl(q_i^\top R_i\,e_3\bigr)\,R_i\,e_3 + \frac{c_s}{2m_Q}\,e_3.
  \label{eq:tension_gradient_full}
\end{equation}
The first term captures the cable-aligned sensitivity: increasing thrust in the direction aligned with the cable direction $q_i$ increases the stretch rate. The factor $q_i^\top R_i\,e_3$ projects the body-$z$ thrust onto the cable axis. The second term is a vertical correction accounting for gravitational coupling. With the control-input gradient~\eqref{eq:tension_gradient_full}, the CBF constraint becomes
\begin{equation}
  \dot{T}_i + \alpha_T\,(T_i - T_{\min}) \geq -\mu_T,
  \label{eq:tautness_cbf}
\end{equation}
which is linear in $f_i$ and can be enforced directly. The tension rate $\dot{T}_i$ is estimated from the measured tension signal (Section~\ref{sec:safety:butterworth}).

\subsubsection{Cable Angle Barrier (Relative Degree 2)}

The cable angle barrier $h_\theta = \cos\theta_{\max} - \cos\theta_i$ involves the cable direction $q_i \in \Sph^2$, whose dynamics~\eqref{eq:swing_dynamics} depend on the quadrotor acceleration (and hence force) through the swing equation. Since $\cos\theta_i = -q_{i,z}$, we have
\begin{equation}
  \dot{h}_\theta = \dot{q}_{i,z} = (\dot{q}_i)_z.
  \label{eq:angle_hdot}
\end{equation}
The cable swing dynamics~\eqref{eq:swing_dynamics} show that $\ddot{q}_i$ depends on the quadrotor acceleration, making $h_\theta$ a \emph{relative-degree-two} constraint with respect to the force input $f_i$. We address this via the HOCBF formulation (Section~\ref{sec:safety:hocbf}).

At the implementation level, the gradient is approximated by the horizontal projection:
\begin{equation}
  \nabla_{f_i} h_\theta \approx -\frac{q_i^{\text{horiz}}}{\norm{q_i^{\text{horiz}}}}, \qquad q_i^{\text{horiz}} = q_i - (q_i^\top e_3)\,e_3,
  \label{eq:angle_gradient}
\end{equation}
which points in the horizontal direction that makes the cable more vertical. Increasing force opposite to the cable's horizontal deviation reduces $\theta_i$.

\subsubsection{Tilt Barrier (Relative Degree 2)}

The tilt barrier $h_{\text{tilt}} = \cos\phi_{\max} - \cos\phi_i$ depends on the attitude $R_i$, which evolves under the torque input $\tau_i$ from Layer~2. Since the safety filter modifies the force $f_i$ (which changes the desired attitude $R_{d_i}$, which in turn changes $\tau_i$), the effective relative degree from $f_i$ to $\cos\phi_i$ is two.

The implementation enforces this constraint by scaling down the horizontal force component when the tilt approaches the limit:
\begin{equation}
  f_i^{\text{horiz}} \leftarrow \eta\,f_i^{\text{horiz}}, \qquad \eta = \max\!\left(0.7,\; 1 - 0.3\,\frac{\max(0, -h_{\text{tilt}})}{0.1}\right)\!,
  \label{eq:tilt_scaling}
\end{equation}
which reduces the horizontal force demand (and thus the required tilt) while preserving vertical thrust for altitude maintenance.

\subsubsection{Swing Rate Barrier (Relative Degree 1)}

The swing rate barrier $h_\omega = \omega_{\max}^2 - \norm{\omega_{q_i}}^2$ is approximately relative-degree-one with respect to force changes that alter the cable angular acceleration. When $h_\omega$ is near violation, the filter blends toward the nominal force:
\begin{equation}
  f_i \leftarrow \beta\,f_i + (1 - \beta)\,f_{\mathrm{nom},i}, \qquad \beta = \max\bigl(0.5,\; 1 + h_\omega\bigr),
  \label{eq:swing_blend}
\end{equation}
which damps rapid force changes that contribute to cable oscillation.

\subsubsection{Collision Avoidance Barrier (Relative Degree 2)}

For the inter-agent barrier $h_{\mathrm{col},ij} = \norm{p_i - p_j}^2 - d_{\min}^2$, the Lie derivative is
\begin{equation}
  \dot{h}_{\mathrm{col},ij} = 2(p_i - p_j)^\top(v_i - v_j),
  \label{eq:collision_hdot}
\end{equation}
and the second derivative involves $\ddot{p}_i - \ddot{p}_j$, which depends on $f_i$ and $f_j$. In the decentralized setting, each agent only controls its own force. Agent $i$ treats $f_j$ as an unknown bounded disturbance and enforces a conservative one-sided constraint with the HOCBF formulation.

% ============================================================
\subsection{Higher-Order CBF for Relative-Degree-Two Constraints}
\label{sec:safety:hocbf}
% ============================================================

For constraints with relative degree two (cable angle, tilt, collision avoidance), we employ the Higher-Order CBF (HOCBF) framework~\cite{xiao2022control}. Define auxiliary functions:
\begin{align}
  \psi_0(x) &= h(x), \label{eq:hocbf_psi0} \\
  \psi_1(x) &= \dot{\psi}_0(x) + \alpha_1\,\psi_0(x), \label{eq:hocbf_psi1}
\end{align}
where $\alpha_1 > 0$. The HOCBF constraint is then
\begin{equation}
  \dot{\psi}_1(x, u) + \alpha_2\,\psi_1(x) \geq 0,
  \label{eq:hocbf_constraint}
\end{equation}
which is now relative-degree-one in $u$ and can be enforced by the QP. The safe set becomes $\mathcal{C}_{\text{HO}} = \{x \mid \psi_0(x) \geq 0\} \cap \{x \mid \psi_1(x) \geq 0\}$. The parameters $\alpha_1, \alpha_2 > 0$ control the convergence rates of the two layers. The forward invariance of $\mathcal{C}_{\text{HO}}$ follows from the cascaded Nagumo condition: $\psi_1 \geq 0$ implies $\dot{\psi}_0 + \alpha_1\,\psi_0 \geq 0$, which in turn implies $\psi_0(t) \geq \psi_0(0)\,e^{-\alpha_1 t} \geq 0$ when $\psi_0(0) \geq 0$.

For the cable angle barrier, the HOCBF is instantiated with $\alpha_1 = \alpha_2 = \alpha_\theta = 2.0$. For the collision avoidance barrier, $\alpha_1 = \alpha_2 = \alpha_{\mathrm{col}} = 3.0$.

% ============================================================
\subsection{Butterworth-Filtered Tension Rate Estimation}
\label{sec:safety:butterworth}
% ============================================================

The tautness CBF constraint~\eqref{eq:tautness_cbf} requires the tension rate $\dot{T}_i$, which is not directly measured. Numerical differentiation of the tension signal amplifies high-frequency sensor noise and the cable vibration modes (which can exceed 100\,Hz for the stiffness values in Table~\ref{tab:sim_params}). To obtain a smooth estimate, we employ a second-order Butterworth low-pass filter.

\subsubsection{Filter Design}

The continuous-time second-order Butterworth transfer function is
\begin{equation}
  H(s) = \frac{\omega_c^2}{s^2 + \sqrt{2}\,\omega_c\,s + \omega_c^2},
  \label{eq:butterworth_tf}
\end{equation}
with cutoff frequency $\omega_c = 2\pi\,f_c$ and $f_c = 15$\,Hz. This filter is discretized via the bilinear (Tustin) transform $s = \frac{2}{T_s}\,\frac{z - 1}{z + 1}$, with frequency pre-warping to correct for the spectral distortion:
\begin{equation}
  \omega_d = \frac{2}{T_s}\,\tan\!\left(\frac{\omega_c\,T_s}{2}\right)\!,
  \label{eq:prewarping}
\end{equation}
where $T_s = 1/f_s$ with $f_s = 200$\,Hz. Defining $K = \omega_d\,T_s / 2$ and $D = 1 + \sqrt{2}\,K + K^2$, the discrete-time difference equation is
\begin{equation}
  y[k] = b_0\,x[k] + b_1\,x[k\!-\!1] + b_2\,x[k\!-\!2] - a_1\,y[k\!-\!1] - a_2\,y[k\!-\!2],
  \label{eq:butterworth_discrete}
\end{equation}
with coefficients
\begin{equation}
  b_0 = b_2 = \frac{K^2}{D}, \quad b_1 = \frac{2K^2}{D}, \quad a_1 = \frac{2(K^2 - 1)}{D}, \quad a_2 = \frac{1 - \sqrt{2}\,K + K^2}{D}.
  \label{eq:butterworth_coefficients}
\end{equation}
The filter input $x[k]$ is the raw finite-difference tension rate $(T[k] - T[k\!-\!1])/T_s$, and the output $y[k]$ is the smoothed estimate $\hat{\dot{T}}_i[k]$.

\subsubsection{Filter Properties}

The Butterworth filter provides maximally flat magnitude response in the passband, zero-ripple gain at DC, and monotonic roll-off at $-40$\,dB/decade beyond $f_c$. For the chosen parameters ($f_c = 15$\,Hz, $f_s = 200$\,Hz), the group delay is approximately $1/(2\pi f_c) \approx 11$\,ms, which is acceptable for the CBF constraint evaluation at 200\,Hz. The filter attenuates the dominant cable vibration frequencies (which scale as $f_{\text{vib}} \sim \frac{1}{2\pi}\sqrt{k_s/m_b} \approx 55$\,Hz for the bead-chain model) by approximately 22\,dB, effectively eliminating high-frequency oscillations from the tension rate estimate while preserving the low-frequency load dynamics that the safety filter must track.

% ============================================================
\subsection{Input-to-State Safety with ESO Coupling}
\label{sec:safety:iss}
% ============================================================

In the presence of the disturbance $w_i(t)$, the nominal CBF condition~\eqref{eq:cbf_condition} may be violated even when the controller satisfies the constraint, because the unmodeled dynamics corrupt the Lie derivative. We employ the Input-to-State Safe CBF (ISSf-CBF) framework~\cite{ames2019control} to provide robust safety guarantees.

\begin{definition}[ISSf-CBF]
\label{def:issf}
A function $h$ is an \emph{Input-to-State Safe CBF} if there exist class-$\mathcal{K}_\infty$ functions $\alpha$ and $\iota$ such that
\begin{equation}
  \sup_{u \in \mathcal{U}} \bigl[L_f h + L_g h\,u\bigr] \geq -\alpha(h) + \iota(\norm{w}),
  \label{eq:issf_condition}
\end{equation}
for all $x \in \mathcal{C}$ and all disturbances $\norm{w} \leq \bar{w}$.
\end{definition}

Under this condition, the system remains in the \emph{inflated} safe set $\mathcal{C}_\mu = \{x \mid h(x) \geq -\mu\}$ where $\mu = \alpha^{-1}(\iota(\bar{w}))$.

\begin{remark}[Practical safety vs.\ hard invariance]
We do \emph{not} claim forward invariance of the nominal safe set $\{h \geq 0\}$. Under the ISSf framework, the system remains in the inflated set $\mathcal{C}_\mu$. Cable angle violations of up to $6.1^\circ$ above $\theta_{\max}$ observed in Section~\ref{sec:results:safety} are consistent with the ISSf bound~\eqref{eq:issf_bound}. The 3.2\% CBF-active time reflects brief excursions into $\mathcal{C}_\mu \setminus \mathcal{C}$, not failures of the safety filter.
\end{remark}

In our implementation, the ESO provides a real-time bound $\norm{\hat{d}_i}$ on the lumped disturbance. The robustness margin implements the ISSf condition with the linear choices $\alpha(r) = \alpha_j\,r$ and $\iota(w) = \kappa_d\,w$:
\begin{equation}
  \dot{h}_j + \alpha_j\,h_j \geq -\mu_{\mathrm{base}} - \kappa_d\,\norm{\hat{d}_i},
  \label{eq:issf_impl}
\end{equation}
where the margin parameters $\mu_{\mathrm{base},j}$ and $\kappa_{d,j}$ are tuned per constraint to match the dimensions of $\dot{h}_j$ (e.g., N/s for tautness, rad/s for angle constraints). The baseline values $\mu_{\mathrm{base}} = 2.0$ and $\kappa_d = 1.5$ are representative. When the ESO reports a large disturbance, the safety margin grows, causing the filter to activate earlier and providing a more conservative (and more robust) control modification.

The steady-state constraint violation bound is
\begin{equation}
  h_j(t) \geq -\frac{\mu_{\mathrm{base}} + \kappa_d\,\bar{d}}{\alpha_j},
  \label{eq:issf_bound}
\end{equation}
which, for the tautness constraint with $\alpha_T = 3.0$ and $\bar{d} = 20$\,m/s$^2$ (the ESO saturation limit), gives a worst-case tension violation of $(2.0 + 1.5 \times 20)/3.0 \approx 10.7$\,N below $T_{\min}$. In practice, the ESO tracks the disturbance much more tightly (Section~\ref{sec:control:layer4}), yielding $\norm{\hat{d}_i} \ll \bar{d}$ and correspondingly tighter safety bounds.

% ============================================================
\subsection{Compatibility with Geometric Attitude Control}
\label{sec:safety:compatibility}
% ============================================================

A critical requirement is that the safety filter does not destabilize the geometric attitude controller (Layer~2). We establish this by showing that the filter's force modification lies within the basin of attraction of the attitude tracking law.

\subsubsection{Force Modification Bound}

Let $f_{\mathrm{nom}}$ and $f_{\mathrm{safe}}$ denote the nominal and filtered force vectors, respectively. The gradient projection modifies the force by
\begin{equation}
  \Delta f = f_{\mathrm{safe}} - f_{\mathrm{nom}} = \lambda^*\,\nabla_f h_j,
  \label{eq:force_modification}
\end{equation}
where $\lambda^* \geq 0$ is the Lagrange multiplier from the projection. This modification changes the desired rotation from $R_{d}^{\mathrm{nom}}$ (constructed from $f_{\mathrm{nom}}$ via~\eqref{eq:b3c_b1d}--\eqref{eq:Rd_assembled}) to $R_{d}^{\mathrm{safe}}$ (constructed from $f_{\mathrm{safe}}$).

\begin{proposition}
\label{prop:attitude_compat}
Let $f_{\mathrm{nom}}$ and $f_{\mathrm{safe}}$ satisfy $\norm{f_{\mathrm{safe}}} \geq f_{\min} > 0$ and the angle between them be $\vartheta = \arccos\!\bigl(f_{\mathrm{nom}}^\top f_{\mathrm{safe}} / (\norm{f_{\mathrm{nom}}}\norm{f_{\mathrm{safe}}})\bigr)$. Then the attitude configuration error satisfies
\begin{equation}
  \Psi_R(R_{d}^{\mathrm{nom}}, R_{d}^{\mathrm{safe}}) \leq 1 - \cos\vartheta.
  \label{eq:attitude_error_bound}
\end{equation}
\end{proposition}

\begin{proof}
The desired rotation $R_d$ is constructed such that its third column is $b_{3_c} = f / \norm{f}$ (cf.~\eqref{eq:b3c_b1d}). When only $b_{3_c}$ changes (holding the yaw angle fixed), the rotation error between the two desired attitudes is a rotation about an axis perpendicular to both $b_{3_c}^{\mathrm{nom}}$ and $b_{3_c}^{\mathrm{safe}}$, with angle $\vartheta$. The trace-based error function gives $\Psi_R = \frac{1}{2}\tr(I - R_{d}^{\mathrm{nom}\top} R_{d}^{\mathrm{safe}}) = 1 - \cos\vartheta$ for a single-axis rotation, establishing the bound.
\end{proof}

\subsubsection{Stability Preservation}

The attitude controller achieves almost-global exponential stability for initial errors $\Psi_R(0) < 2$ (Section~\ref{sec:control:layer2}). For compatibility, we require that the CBF-induced attitude perturbation satisfies $\Psi_R(R_d^{\mathrm{nom}}, R_d^{\mathrm{safe}}) < 2 - \Psi_R(R_i, R_d^{\mathrm{nom}})$, ensuring the composite error remains in the stability region.

\begin{theorem}[Safety-Stability Compatibility]
\label{thm:compatibility}
Suppose the tilt barrier $h_{\mathrm{tilt}}$ with $\phi_{\max} = 0.5$\,rad is enforced. Then the attitude error between the nominal and safe desired rotations satisfies $\Psi_R(R_d^{\mathrm{nom}}, R_d^{\mathrm{safe}}) < 1$, and the geometric controller retains almost-global exponential stability.
\end{theorem}

\begin{proof}
The tilt constraint ensures $\phi_i \leq \phi_{\max} = 0.5$\,rad, which bounds the angle between the body $z$-axis and vertical. Since the safety filter modifies only the force direction (not magnitude arbitrarily), and the tilt constraint~\eqref{eq:tilt_scaling} limits the horizontal force component, the angle between $f_{\mathrm{nom}}$ and $f_{\mathrm{safe}}$ is bounded by $\vartheta \leq 2\phi_{\max} = 1.0$\,rad. By Proposition~\ref{prop:attitude_compat}, $\Psi_R(R_d^{\mathrm{nom}}, R_d^{\mathrm{safe}}) \leq 1 - \cos(1.0) \approx 0.46 < 1$. Since the attitude tracking error $\Psi_R(R_i, R_d^{\mathrm{nom}})$ converges to zero exponentially, the subadditivity property of $\Psi_R$ for rotations satisfying $\Psi_R < 2$ (cf.~\cite{lee2010geometric}, Proposition~3), which holds in the almost-global stability region, gives
\begin{equation}
  \Psi_R(R_i, R_d^{\mathrm{safe}}) \leq \Psi_R(R_i, R_d^{\mathrm{nom}}) + \Psi_R(R_d^{\mathrm{nom}}, R_d^{\mathrm{safe}}) < 0 + 0.46 + \varepsilon < 2,
  \label{eq:composite_error}
\end{equation}
for sufficiently small $\varepsilon$, provided the initial attitude error is bounded (which holds after the transient settling time $t_s \approx 5/\omega_o$). Thus the composite system remains in the almost-global stability region $\{\Psi_R < 2\}$.
\end{proof}

\subsubsection{Timescale Separation Argument}

The compatibility result relies on the timescale separation between the attitude loop (200\,Hz) and the safety filter (200\,Hz, but producing slowly varying force modifications). The CBF projection~\eqref{eq:force_modification} produces a continuous modification of the desired attitude, which the inner attitude loop tracks within its convergence envelope. Since the attitude loop bandwidth ($k_R/J \approx 200$\,rad/s) exceeds the safety filter's effective bandwidth (bounded by the Butterworth cutoff at $2\pi \times 15 \approx 94$\,rad/s), the attitude controller can track the filtered desired attitude without losing stability. Standard singular perturbation analysis~\cite{khalil2002nonlinear} formalizes this argument: the fast attitude dynamics converge to a boundary layer around $R_d^{\mathrm{safe}}$ on the timescale $J/k_\Omega \approx 5$\,ms, while the force modification evolves on the slow timescale of the position and cable dynamics ($\sim 50$--$200$\,ms).

% ============================================================
\subsection{Constraint Priority and Feasibility}
\label{sec:safety:feasibility}
% ============================================================

The safety filter solves, at each control cycle, the quadratic program
\begin{equation}
  f_{\mathrm{safe}} = \argmin_{f \in \R^3}\; \norm{f - f_{\mathrm{nom}}}^2 + \lambda\sum_j \delta_j^2 \quad
  \text{s.t.}\;\; \dot{h}_j + \alpha_j\,h_j \geq -\mu_j - \delta_j,\;\; \delta_j \geq 0,
  \label{eq:cbf_qp}
\end{equation}
where $\lambda = 100$ penalizes constraint relaxation and $\mu_j$ incorporates the ISSf margin~\eqref{eq:issf_impl}.

\paragraph{Implementation} While~\eqref{eq:cbf_qp} describes the ideal multi-constraint optimization, our decentralized implementation uses sequential gradient projection, which processes constraints in priority order. For our 6-constraint problem, this is equivalent to the QP when at most one constraint is active (the typical case: the CBF is active only 3.2\% of the time). When multiple constraints activate simultaneously, the sequential method may produce a suboptimal solution; this approximation is tight when constraints activate sequentially, which is the common case in our experiments. The slack variables $\delta_j$ provide soft relaxation, and the sequential projection imposes an implicit priority ordering:

\begin{enumerate}
  \item Cable tautness (lower bound): highest priority, as loss of cable tension leads to uncontrolled swing on re-engagement.
  \item Cable tautness (upper bound): prevents structural overload.
  \item Cable angle: maintains geometric assumptions of the controller.
  \item Quadrotor tilt: prevents actuator saturation and maintains controllability.
  \item Swing rate: secondary damping constraint.
  \item Collision avoidance: inter-agent safety.
\end{enumerate}

This ordering ensures that tautness---the most critical constraint for cable-suspended transport---is always enforced first, and subsequent constraints are satisfied to the extent compatible with the higher-priority constraints. The tilt constraint~\eqref{eq:tilt_scaling} uses a multiplicative scaling rather than additive projection, preventing it from conflicting with the vertical force component needed for altitude maintenance.

\begin{remark}[Collision avoidance geometry]
For the $N = 3$ triangular formation with $r_f = 0.6$\,m and $d_{\min} = 0.8$\,m, the formation geometry guarantees $\norm{p_i - p_j} \geq 2r_f\sin(\pi/N) = 1.04$\,m $> d_{\min}$ at equilibrium, providing a 0.24\,m margin. Collision avoidance thus serves as a backup for transient perturbations, not a primary safety mechanism. For formations with $N \geq 4$ quadrotors, the collision avoidance constraints may create pairwise conflicts (agent $i$ must move away from agent $j$, but the required direction violates tautness for cable $i$). In such cases, the sequential projection defers collision avoidance in favor of tautness, and the collision constraint is recovered through the formation controller's inherent separation maintenance. A full multi-agent CBF formulation using consensus-based constraint sharing~\cite{ames2019control} would resolve such conflicts optimally but requires inter-agent communication, which is beyond the decentralized scope of this work.
\end{remark}

% ============================================================
\subsection{Computational Complexity}
\label{sec:safety:complexity}
% ============================================================

The sequential gradient projection---an approximation to the QP~\eqref{eq:cbf_qp}---has per-agent complexity $\mathcal{O}(N_c)$ where $N_c = 6$ is the number of constraints, making it suitable for real-time execution at 200\,Hz. Each projection step~\eqref{eq:force_modification} requires one dot product and one vector addition ($\mathcal{O}(1)$ operations in $\R^3$). The Butterworth filter~\eqref{eq:butterworth_discrete} adds $\mathcal{O}(1)$ operations per timestep. The barrier function evaluation~\eqref{eq:cbf_tautness}--\eqref{eq:cbf_collision} requires the drone state, cable state, and tension, all of which are already computed by the control and estimation pipelines.

The total per-agent computational cost of the safety filter is dominated by the tension gradient computation~\eqref{eq:tension_gradient_full}, which requires one rotation matrix--vector product and one dot product, for a total of approximately 30 floating-point operations per constraint evaluation. This is negligible compared to the ESKF update ($\sim 2000$ FLOPs) and the Drake simulation step.

% ============================================================
\subsection{Performance Impact and Chattering Analysis}
\label{sec:safety:tradeoffs}
% ============================================================

Any safety filter necessarily sacrifices some tracking performance when it intervenes: the filtered force $f_{\mathrm{safe}}$ deviates from the nominal $f_{\mathrm{nom}}$, introducing a persistent tracking error proportional to the projection magnitude~\eqref{eq:force_modification}. In our experiments (Section~\ref{sec:results:safety}), the CBF is active for only 3.2\% of the simulation time, concentrated in smooth bursts during aggressive cornering. The tracking RMSE increases by less than 2\% when the CBF is active versus inactive (24.2\,cm vs.\ 23.8\,cm), indicating that the safety overlay imposes minimal performance cost under normal operation.

Three mechanisms mitigate the chattering that would otherwise arise from switching between constrained and unconstrained regimes:
\begin{enumerate}
  \item \emph{Butterworth tension-rate filter} (Section~\ref{sec:safety:butterworth}): smooths the noisy tension derivative input, preventing high-frequency constraint activation/deactivation cycles.
  \item \emph{ISSf margin buffer}: the robustness margin~\eqref{eq:issf_impl} creates a soft boundary layer around each constraint, so the filter activates gradually (proportional to the margin violation) rather than abruptly.
  \item \emph{Timescale separation}: the safety filter's effective bandwidth ($f_c = 15$\,Hz) is well below the attitude controller bandwidth ($\sim$$200$\,Hz), ensuring that CBF-induced force changes are tracked smoothly by the inner loop without exciting oscillatory modes.
\end{enumerate}
The net effect is that the safety filter produces continuous, slowly varying force modifications that the attitude controller can track within its convergence envelope (Theorem~\ref{thm:compatibility}), avoiding the bang-bang behavior often associated with barrier function enforcement.