% !TEX root = ../Main.tex
This paper presented GPAC, a four-layer hierarchical controller for decentralized cooperative aerial transport on $\SEthree \times (\Sph^2)^N$. The architecture's central property is that each quadrotor runs an identical control stack using only local sensors and its cable, yet the collective system achieves coordinated payload transport with formal stability and safety certificates. No knowledge of the team size $N$ or payload mass $m_L$ is required, and no payload, cable, or adaptive states are exchanged between agents.

The hazard-oriented decomposition---each layer targeting a specific failure mode (Table~\ref{tab:failure_modes})---enables three properties that are individually well-studied but rarely combined: geometric manifold-based stability, decentralized adaptive learning, and runtime constraint enforcement. Hierarchical timescale separation limits cross-layer fault propagation, full decentralization eliminates single points of failure, and per-agent policy is team-size-invariant by construction. High-fidelity simulation with flexible cables, onboard sensor fusion, and wind turbulence confirms that these theoretical properties translate to practical tracking accuracy at sub-MFLOP computational cost.

The present results are simulation-only, making hardware flight experiments the immediate next step. Two architectural limitations motivate further algorithmic development: the decentralized load estimator exhibits a 4 times accuracy gap relative to the centralized baseline, and the sequential CBF projection guarantees only priority-ordered feasibility—rather than full ISSf—when multiple constraints activate simultaneously. Accordingly, future work targets distributed consensus-based estimation to narrow the gap with centralized methods while preserving communication efficiency.
