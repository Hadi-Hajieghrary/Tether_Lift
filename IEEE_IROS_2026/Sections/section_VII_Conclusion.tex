% !TEX root = ../Main.tex
This paper presented GPAC, a four-layer hierarchical controller for decentralized cooperative aerial transport on $\SEthree \times (\Sph^2)^N$. The architecture's central property is that each quadrotor runs an identical control stack using only local sensors and its cable, yet the collective system achieves coordinated payload transport with formal stability and safety certificates. No knowledge of the team size $N$ or payload mass $m_L$ is required, and no payload, cable, or adaptive states are exchanged---only neighbor GPS positions for collision avoidance.

The hazard-oriented decomposition---each layer targeting a specific failure mode (Table~\ref{tab:failure_modes})---enables three properties that are individually well-studied but rarely combined: geometric manifold-based stability, decentralized adaptive learning, and runtime constraint enforcement. Hierarchical timescale separation limits cross-layer fault propagation, full decentralization eliminates single points of failure, and per-agent policy is team-size-invariant by construction. High-fidelity simulation with flexible cables, onboard sensor fusion, and wind turbulence confirms that these theoretical properties translate to practical tracking accuracy at sub-MFLOP computational cost.

\textit{Limitations.} Results are simulation-only; hardware validation is the immediate next step. The shared broadcast trajectory (including $\psi_d$) is a single point of failure, and neighbor GPS exchange at 10\,Hz for collision avoidance constitutes minimal state sharing. The decentralized load estimator shows a $4\times$ gap versus the centralized baseline, motivating distributed consensus-based estimation. Agent dropout and dynamic reconfiguration are unaddressed. Sequential CBF projection guarantees only priority-ordered feasibility (not ISSf) under simultaneous multi-constraint activation.

\textit{Future work.} Flight experiments with 3--6 quadrotors, distributed estimation to narrow the centralized gap while preserving efficiency, agent dropout detection with $N$-adaptive re-estimation, and full QP-based multi-constraint resolution.
