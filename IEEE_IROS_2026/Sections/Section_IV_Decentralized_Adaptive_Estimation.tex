A defining feature of the proposed architecture is that each quadrotor estimates the system state using only locally available information---its own sensors and the cable connecting it to the payload---without explicit inter-agent communication. Intuitively, each drone asks a single question: ``how heavy is my share of the payload?'' It answers by observing its cable tension and angle---when the cable is taut, the vertical component of the tension divided by the sensed acceleration magnitude directly reveals the per-drone mass share $\theta_i = m_L / N$. Past observations are stored and replayed so that even during hover, when excitation is minimal, the answer keeps improving.

This section presents three estimation subsystems: (i)~an Error-State Kalman Filter (ESKF) for onboard navigation, (ii)~a geometric load-state estimator derived from cable kinematics, and (iii)~an adaptive payload-mass estimator based on concurrent learning.

% ============================================================
\subsection{Per-Drone Navigation: Error-State Kalman Filter}
\label{sec:estimation:eskf}
% ============================================================

The ESKF introduced in Section~\ref{sec:modeling:eskf} provides each quadrotor $i$ with a fused estimate of its own position $\hat{p}_i$, velocity $\hat{v}_i$, attitude $\hat{R}_i$, and sensor biases $\hat{b}_{a_i}$, $\hat{b}_{g_i}$ from the onboard IMU~(200\,Hz), GPS~(10\,Hz), and barometer~(25\,Hz). Since the ESKF formulation was detailed in Section~\ref{sec:modeling:eskf}, we focus here on two aspects relevant to the decentralized estimation pipeline.

\subsubsection{Interface with Control and Estimation Layers}

The ESKF provides the following quantities to downstream modules:
\begin{itemize}
  \item \emph{Position and velocity estimates} $(\hat{p}_i, \hat{v}_i)$: used by the GPAC position controller (Layer~1, Section~\ref{sec:control:layer1}), the ESO (Layer~4, Section~\ref{sec:control:layer4}), and the decentralized load estimator (Section~\ref{sec:estimation:load}).
  \item \emph{Attitude estimate} $\hat{R}_i$: used by the geometric attitude controller (Layer~2, Section~\ref{sec:control:layer2}).
  \item \emph{Covariance diagonal} $\diag(P_i)$: used by the load estimator to weight the cable-based measurement update.
\end{itemize}
When the estimated state port is connected to the GPAC controller, the controller uses $(\hat{p}_i, \hat{v}_i)$ in place of the plant-state position and velocity, closing the loop through the estimator rather than through ground truth. This architecture ensures that all control and estimation decisions are based solely on sensor-derived information.

\subsubsection{GPS Dropout Resilience}

The GPS sensor model (Section~\ref{sec:modeling:sensors}) includes a configurable dropout probability $p_{\mathrm{drop}}$. During GPS outages, the ESKF prediction step~\eqref{eq:eskf_prop_p}--\eqref{eq:eskf_prop_q} continues to propagate the state using IMU data, and the covariance grows according to~\eqref{eq:eskf_cov_prop}. The barometer update provides altitude observability at 25\,Hz even when GPS is unavailable, preventing unbounded vertical drift. Horizontal drift during outages is bounded by the accelerometer bias stability ($\sigma_{b_a} = 10^{-3}$\,m/s$^2$, time constant $\tau_a = 3600$\,s), yielding worst-case position error growth of approximately $\sigma_{b_a}\,t^2/2$.

% ============================================================
\subsection{Decentralized Load-State Estimation}
\label{sec:estimation:load}
% ============================================================

Each quadrotor estimates the payload position $\hat{p}_{L,i}$ and velocity $\hat{v}_{L,i}$ from its own position estimate and the cable geometric relationship, without communicating with the other agents. This provides the reference signal for the trajectory tracking and the regressor for the adaptive mass estimator.

\subsubsection{Geometric Measurement Model}

When the cable connecting quadrotor $i$ to the payload is taut, the payload position can be computed geometrically. Let $n_i \in \Sph^2$ denote the unit vector pointing from the quadrotor toward the load attachment point, and let $L_i$ denote the cable rest length. The geometric measurement is
\begin{equation}
  z_{p_i} = \hat{p}_i - L_i\,n_i,
  \label{eq:load_geometric_meas}
\end{equation}
which estimates $p_L + R_L\,\rho_i^L$ (the world-frame attachment point on the payload). For the current system with attachment points near the payload center of mass, $R_L\,\rho_i^L \approx p_L$, and~\eqref{eq:load_geometric_meas} directly estimates the load position.

The quality of this geometric estimate depends on (i)~the accuracy of $\hat{p}_i$ from the ESKF, (ii)~the cable direction measurement noise, and (iii)~the cable sag, which introduces a systematic bias proportional to the catenary departure from a straight line. Under tension $T_i$, the sag is $\delta_{\text{sag}} \approx m_{\text{rope}}\,g\,L_i/(8T_i)$, which is small when the cable is taut.

\subsubsection{Kalman Filter Formulation}

Each drone maintains a discrete-time Kalman filter with state $\hat{x}_{L,i} = [\hat{p}_{L,i}^\top,\, \hat{v}_{L,i}^\top]^\top \in \R^6$ and diagonal covariance $P_i \in \R^{6 \times 6}$ (stored as a 6-vector for efficiency).

\paragraph{Prediction Step} The load dynamics are modeled as a constant-velocity process:
\begin{align}
  \hat{p}_{L,i}^{-} &= \hat{p}_{L,i} + \hat{v}_{L,i}\,\Delta t, \label{eq:load_predict_p} \\
  \hat{v}_{L,i}^{-} &= (1 - \beta_v)\,\hat{v}_{L,i} + \beta_v\,\hat{v}_i, \label{eq:load_predict_v}
\end{align}
where $\beta_v = 0.05$ is a velocity damping factor that gently biases the load velocity estimate toward the quadrotor velocity (a quasi-static assumption valid during slow transport). The predicted covariance grows as
\begin{equation}
  P_i^{-} = P_i + \diag\!\bigl(\sigma_{Q_p}^2\,\Delta t,\; \sigma_{Q_v}^2\,\Delta t\bigr),
  \label{eq:load_predict_cov}
\end{equation}
with position process noise $\sigma_{Q_p} = 0.05$\,m and velocity process noise $\sigma_{Q_v} = 0.2$\,m/s.

\paragraph{Measurement Update} The geometric measurement~\eqref{eq:load_geometric_meas} is fused via a scalar Kalman gain per axis:
\begin{equation}
  K_{i,k} = \frac{\kappa_c\,P_{i,k}^{-}}{P_{i,k}^{-} + R_k}, \qquad k \in \{x, y, z\},
  \label{eq:load_kalman_gain}
\end{equation}
where $\kappa_c = 0.9$ is the cable trust factor and the measurement variance is
\begin{equation}
  R_k = \sigma_{\text{cat}}^2 + \sigma_n^2\,L_i^2 + \sigma_L^2,
  \label{eq:load_meas_variance}
\end{equation}
composed of catenary model uncertainty ($\sigma_{\text{cat}} = 0.15$\,m), cable direction noise ($\sigma_n = 0.02$\,rad, scaled by cable length), and length uncertainty ($\sigma_L = 0.01$\,m). The measurement noise is modulated by the cable tension via a confidence factor:
\begin{equation}
  \xi_T = \min\!\left(1,\; \frac{T_i}{T_{\text{conf}}}\right)\!, \qquad R_k \leftarrow \frac{R_k}{0.1 + 0.9\,\xi_T},
  \label{eq:tension_confidence}
\end{equation}
with $T_{\text{conf}} = 20$\,N. When the cable is slack ($T_i \approx 0$), the measurement variance increases dramatically and the filter relies primarily on the prediction.

\paragraph{Outlier Rejection} The innovation $\nu_i = z_{p_i} - \hat{p}_{L,i}^{-}$ is validated against the expected standard deviation:
\begin{equation}
  \norm{\nu_i} < \kappa_\nu\,\sqrt{\tfrac{1}{3}\,\mathbf{1}^\top P_i^{-} + R_k}, \qquad \kappa_\nu = 3.0.
  \label{eq:outlier_reject}
\end{equation}
Measurements exceeding this gate are rejected, preventing cable whip or GPS outliers from corrupting the load estimate.

\paragraph{State and Covariance Update} If the measurement passes validation:
\begin{align}
  \hat{p}_{L,i,k} &\leftarrow \hat{p}_{L,i,k}^{-} + K_{i,k}\,\nu_{i,k}, \label{eq:load_update_p} \\
  P_{i,k} &\leftarrow (1 - K_{i,k})\,P_{i,k}^{-}, \label{eq:load_update_cov}
\end{align}
for each axis $k$. The velocity estimate is additionally blended with the innovation-implied velocity:
\begin{equation}
  \hat{v}_{L,i} \leftarrow (1 - \alpha_v)\,\hat{v}_{L,i}^{-} + \alpha_v\,\frac{\nu_i}{\Delta t},
  \label{eq:load_vel_blend}
\end{equation}
with blending factor $\alpha_v = 0.3$. The covariance is bounded element-wise to $[10^{-6}, 100]$ to prevent numerical issues.

% ============================================================
\subsection{Centralized Load Estimator with Cable Constraints}
\label{sec:estimation:centralized}
% ============================================================

As a comparison baseline, we also implement a centralized EKF that fuses all available measurements. Although this estimator requires a central processor with access to all quadrotor states, it provides a performance upper bound for evaluating the decentralized approach.

\subsubsection{State and Process Model}

The centralized estimator maintains the same state $\hat{x}_L = [\hat{p}_L^\top,\, \hat{v}_L^\top]^\top \in \R^6$ with the constant-velocity transition model:
\begin{equation}
  \hat{x}_L^{-} = F\,\hat{x}_L, \qquad F = \begin{bmatrix} I_3 & \Delta t\,I_3 \\ 0 & I_3 \end{bmatrix}\!,
  \label{eq:cent_predict}
\end{equation}
and the full $6 \times 6$ covariance propagation $P^{-} = F\,P\,F^\top + Q$ with process noise $Q = \Delta t\,\diag(\sigma_{Q_p}^2\,I_3,\;\sigma_{Q_v}^2\,I_3)$.

\subsubsection{GPS Measurement Update}

When a load GPS measurement $\tilde{p}_L^{\text{GPS}}$ is available, the standard Kalman update is applied:
\begin{align}
  y &= \tilde{p}_L^{\text{GPS}} - H_{\text{GPS}}\,\hat{x}_L^{-}, \label{eq:cent_gps_innov} \\
  S &= H_{\text{GPS}}\,P^{-}\,H_{\text{GPS}}^\top + R_{\text{GPS}}, \label{eq:cent_gps_S} \\
  K &= P^{-}\,H_{\text{GPS}}^\top\,S^{-1}, \label{eq:cent_gps_K} \\
  \hat{x}_L &\leftarrow \hat{x}_L^{-} + K\,y, \label{eq:cent_gps_x} \\
  P &\leftarrow (I - K\,H_{\text{GPS}})\,P^{-}, \label{eq:cent_gps_P}
\end{align}
where $H_{\text{GPS}} = [I_3,\; 0_{3 \times 3}]$ and $R_{\text{GPS}} = \diag(\sigma_{xy}^2,\; \sigma_{xy}^2,\; \sigma_z^2)$.

\subsubsection{Taut-Gated Cable Constraint Updates}

The key advantage of the centralized estimator is its ability to fuse \emph{cable-range constraints} from all $N$ quadrotors simultaneously. For each cable $i$ with measured tension $T_i \geq T_{\text{taut}} = 1.0$\,N (indicating tautness---deliberately set below the CBF enforcement threshold $T_{\min} = 2.0$\,N so that the estimator begins incorporating cable measurements before the safety filter becomes active, providing earlier parameter convergence), the constraint is
\begin{equation}
  h_i(\hat{x}_L) = \norm{\hat{p}_L - p_{Q_i}} - L_i = 0,
  \label{eq:cent_cable_constraint}
\end{equation}
where $p_{Q_i}$ is the estimated attachment point on quadrotor $i$. This nonlinear constraint is linearized via the Jacobian:
\begin{equation}
  H_{c_i} = \frac{1}{\norm{\hat{p}_L - p_{Q_i}}}\begin{bmatrix} (\hat{p}_L - p_{Q_i})^\top & 0_{1 \times 3} \end{bmatrix}\!,
  \label{eq:cable_jacobian}
\end{equation}
and a scalar EKF update is applied with innovation $y_i = -h_i(\hat{x}_L)$ and measurement noise $R_{c_i} = \sigma_c^2$ with $\sigma_c = 0.02$\,m:
\begin{align}
  S_i &= H_{c_i}\,P\,H_{c_i}^\top + R_{c_i}, \label{eq:cable_S} \\
  K_i &= P\,H_{c_i}^\top / S_i, \label{eq:cable_K} \\
  \hat{x}_L &\leftarrow \hat{x}_L + K_i\,y_i, \qquad P \leftarrow (I - K_i\,H_{c_i})\,P. \label{eq:cable_update}
\end{align}
The cable constraints are applied sequentially for each taut cable after the GPS update. When the cable is slack ($T_i < T_{\text{taut}}$), the corresponding constraint is gated off, preventing the estimator from being corrupted by slack cable geometry.

This taut-gating mechanism is essential: without it, slack cables would inject spurious position information, as the cable endpoint position bears no geometric relationship to the load when the cable is not under tension.

% ============================================================
\subsection{Adaptive Payload Mass Estimation}
\label{sec:estimation:adaptive}
% ============================================================

Each quadrotor estimates its share of the payload mass $\theta_i \approx m_L / N$ using a concurrent-learning adaptive law that requires only locally available measurements: cable tension, cable direction, and the estimated load acceleration. This estimate feeds into the tension feedforward and trajectory generation modules.

\subsubsection{Regressor Derivation}

Consider the load translational dynamics~\eqref{eq:payload_translational}. In the quasi-static regime (small $\ddot{p}_L$) with $N$ cables in a symmetric configuration, the per-cable equilibrium yields
\begin{equation}
  T_i\,\cos\phi_i \approx \frac{m_L}{N}\,\norm{g\,e_3 + \hat{a}_L},
  \label{eq:tension_equilibrium}
\end{equation}
where $\phi_i = \arccos(-n_{i,z})$ is the cable angle from vertical (with $n_i$ pointing from quadrotor to load) and $\hat{a}_L$ is the estimated load acceleration. Defining the per-drone parameter $\theta = m_L / N$, the regressor $Y_i$ and measurement $\varphi_i$ are:
\begin{equation}
  Y_i = \norm{\hat{a}_L + g\,e_3}, \qquad \varphi_i = T_i\,\cos\phi_i,
  \label{eq:regressor_def}
\end{equation}
such that the parametric model $\varphi_i = Y_i\,\theta + \varepsilon_i$ holds with error $\varepsilon_i$ arising from non-equilibrium dynamics, cable sag, and asymmetric load distribution. This approximation holds when the cable is nearly vertical ($\phi_i \ll 1$) and the payload acceleration is dominated by the vertical component. Under these conditions, $\cos\phi_i \approx 1$ and $\norm{g\,e_3 + a_L} \approx |g + a_{L,z}|$, recovering the exact vertical force balance. For non-vertical cables during aggressive maneuvering, the approximation introduces a modeling error $\varepsilon_i$ that is absorbed into the UUB convergence bound below.

\subsubsection{Acceleration Estimation}

The load acceleration is not directly measured; it is estimated from the load velocity provided by the decentralized load estimator (Section~\ref{sec:estimation:load}) via numerical differentiation with a first-order low-pass filter:
\begin{equation}
  \hat{a}_L[k] = \alpha_f\,\frac{\hat{v}_{L,i}[k] - \hat{v}_{L,i}[k-1]}{\Delta t} + (1 - \alpha_f)\,\hat{a}_L[k-1],
  \label{eq:accel_filter}
\end{equation}
where $\alpha_f = \Delta t / (\tau_f + \Delta t)$ with filter time constant $\tau_f = 0.1$\,s. This filter suppresses the differentiation noise while preserving the quasi-static acceleration signal needed for the regressor.

\subsubsection{Concurrent Learning Adaptation Law}

The sliding variable for the adaptive law is defined as
\begin{equation}
  s_i = \dot{e}_{L,i} + \lambda\,e_{L,i}, \qquad e_{L,i} = \hat{p}_{L,i} - p_{d_L},
  \label{eq:sliding_variable}
\end{equation}
where $p_{d_L}$ is the desired load position and $\lambda = 1.0$ is the filter gain. The projection of $s_i$ onto the cable direction yields a scalar error $s_{\text{proj}} = s_i^\top n_i$ that isolates the mass-dependent component of the tracking error.

The concurrent learning update law is~\cite{chowdhary2010concurrent}
\begin{equation}
  \dot{\hat{\theta}}_i = -\gamma\,Y_i\,s_{\text{proj}} - \gamma\,\rho\sum_{j=1}^{M_i} Y_j\,(Y_j\,\hat{\theta}_i - \varphi_j),
  \label{eq:adaptive_update}
\end{equation}
where $\gamma = 0.5$ is the adaptation gain and $\rho = 0.5$ weights the concurrent learning term. The first term is the standard gradient descent driven by the current tracking error; the second term replays the stored data pairs $\{(Y_j, \varphi_j)\}_{j=1}^{M_i}$ from the history buffer.

\subsubsection{History Buffer Management}

Each quadrotor maintains a history buffer of up to $\bar{M} = 50$ data points $\{(Y_j, \varphi_j, t_j)\}$. New data points are admitted only when the excitation level is sufficient ($Y_{\mathrm{new}} > Y_{\min} = 0.5$\,m/s$^2$, indicating non-hovering conditions) and when the new regressor value differs from the buffer mean by at least $\delta_Y = 0.1$:
\begin{equation}
  \abs{Y_{\mathrm{new}} - \bar{Y}} > \delta_Y, \qquad \bar{Y} = \frac{1}{M_i}\sum_{j=1}^{M_i} Y_j.
  \label{eq:data_admission}
\end{equation}
This simplified information criterion (compared to the SVD-based rank check proposed in~\cite{chowdhary2013exponentially}) reduces the per-step computational cost while still ensuring diversity in the stored data.

\subsubsection{Parameter Projection}

After each discrete integration step, the estimate is projected to $[\theta_{\min}, \theta_{\max}] = [0.1, 50.0]$\,kg:
\begin{equation}
  \hat{\theta}_i \leftarrow \mathrm{proj}_{[\theta_{\min},\,\theta_{\max}]}(\hat{\theta}_i).
  \label{eq:adaptive_projection}
\end{equation}
The feedforward output $\hat{W}_i = \hat{\theta}_i\,g$ (estimated weight share in Newtons) is provided to the position controller for gravity compensation.

\subsubsection{Convergence Analysis}

For a scalar parametric model $\varphi = Y\theta$ with concurrent learning, the parameter error $\tilde{\theta} = \hat{\theta} - \theta$ satisfies~\cite{chowdhary2013exponentially}
\begin{equation}
  \dot{V}_\theta = \tilde{\theta}\,\dot{\tilde{\theta}} = -\gamma\,Y^2\,\tilde{\theta}^2 - \gamma\,\rho\,\tilde{\theta}\!\sum_j Y_j^2\,\tilde{\theta} \leq -\gamma\,(\bar{Y}^2 + \rho\,\Sigma_Y)\,\tilde{\theta}^2,
  \label{eq:convergence_lyapunov}
\end{equation}
where $V_\theta = \tilde{\theta}^2 / (2\gamma)$ and $\Sigma_Y = \sum_j Y_j^2 / M$ is the average squared regressor over the history. Provided $\Sigma_Y > 0$ (which holds whenever the history buffer contains at least one data point with $Y_j \neq 0$), the convergence rate is exponential:
\begin{equation}
  \abs{\tilde{\theta}(t)} \leq \abs{\tilde{\theta}(0)}\,\exp\!\bigl(-\gamma\,\rho\,\Sigma_Y\,t\bigr),
  \label{eq:exponential_convergence}
\end{equation}
independently of whether the online excitation $Y(t)$ is persistently exciting. In practice, the regressor model~\eqref{eq:tension_equilibrium} is approximate due to cable flexibility, payload rotation, and acceleration estimation noise. Under bounded modeling error $|\varepsilon_i| \leq \bar{\varepsilon}$, the convergence guarantee weakens to \emph{uniformly ultimately bounded} (UUB): $|\tilde{\theta}(t)| \leq \max\{|\tilde{\theta}(0)|\,e^{-\gamma\rho\Sigma_Y t},\; \bar{\varepsilon}/(\rho\Sigma_Y)\}$, where the ultimate bound depends on the regressor signal-to-noise ratio. This is the key advantage of concurrent learning: parameter convergence is guaranteed once the history buffer accumulates sufficiently informative data, even during hovering phases when $Y(t) \approx g$ is nearly constant.

\begin{remark}[Non-trivial adaptations for decentralized geometric transport]
Three aspects distinguish the present estimator from the standard concurrent learning formulation in~\cite{chowdhary2010concurrent, chowdhary2013exponentially}:
(i)~the scalar regressor~\eqref{eq:regressor_def} is extracted from the coupled multi-body dynamics via the local cable tension and direction, avoiding the need for a full system model or inter-agent state exchange;
(ii)~the tension-confidence factor~\eqref{eq:tension_confidence} modulates the measurement noise during slack-to-taut transitions, preventing the estimator from being corrupted by transient cable dynamics;
(iii)~the convergence guarantee~\eqref{eq:exponential_convergence} holds independently per agent, yet the implicit coordination result (Section~\ref{sec:control:layer1}) ensures that the sum $\sum_i \hat{\theta}_i \to m_L$ as each $\hat{\theta}_i \to m_L/N$, providing correct collective force without any consensus protocol.
\end{remark}

% ============================================================
\subsection{Estimation Architecture Integration}
\label{sec:estimation:integration}
% ============================================================

Fig.~\ref{fig:estimation_architecture} summarizes the information flow in the decentralized estimation pipeline. Each quadrotor runs three estimators in cascade at the rates indicated:

\begin{figure}[t]
  \centering
  \fbox{\parbox{0.92\columnwidth}{\centering\vspace{0.3em}
    {\small
    \textbf{IMU/GPS/Baro} $\to$ \textbf{ESKF} (200\,Hz) \\[-0.2em]
    $\downarrow$ \; $\hat{p}_i,\, \hat{v}_i,\, \hat{R}_i$ \\[0.1em]
    \textbf{Cable direction + length} $\to$ \textbf{Load KF} (50\,Hz) \\[-0.2em]
    $\downarrow$ \; $\hat{p}_{L,i},\, \hat{v}_{L,i}$ \\[0.1em]
    \textbf{Cable tension} $\to$ \textbf{Mass Estimator} (50\,Hz) \\[-0.2em]
    $\downarrow$ \; $\hat{\theta}_i$ (to Layer~1 tension feedforward)
    }
  \vspace{0.3em}}}
  \caption{Decentralized estimation pipeline for a single quadrotor. Information flows strictly downward (bottom estimators depend on upper estimators) with no inter-agent communication. The only shared information is the common desired trajectory $p_{d_L}(t)$.}
  \label{fig:estimation_architecture}
\end{figure}

\begin{enumerate}
  \item \textbf{ESKF} (200\,Hz): fuses IMU, GPS, and barometer to produce the quadrotor navigation solution. This is the innermost and fastest estimator.
  \item \textbf{Load state KF} (50\,Hz): uses the ESKF output and the cable geometric relationship to estimate the payload position and velocity.
  \item \textbf{Adaptive mass estimator} (50\,Hz): uses the load state estimate, cable tension, and cable direction to estimate $\theta_i = m_L / N$ via concurrent learning.
\end{enumerate}

The cascade structure ensures that estimation errors propagate in one direction (upstream errors affect downstream estimates, but not vice versa), simplifying the stability analysis. The separation of timescales---navigation at 200\,Hz versus load/mass estimation at 50\,Hz---allows each layer to treat its inputs as quasi-static over its own update interval.

\paragraph{Comparison with Centralized Estimation}

The decentralized approach trades optimality for scalability and robustness. The centralized load EKF~(Section~\ref{sec:estimation:centralized}) can fuse $N$ cable constraints simultaneously, achieving smaller estimation error (as demonstrated in Section~\ref{sec:results}), but requires all quadrotor states to be transmitted to a central node at each update cycle. The decentralized estimator, by contrast, requires no communication, tolerates the loss of any agent's data without affecting the others, and scales to arbitrary team sizes without increasing the per-agent computational burden.

The estimation error bound for the decentralized case can be related to the centralized optimum via
\begin{equation}
  \mathbb{E}\!\bigl[\norm{\hat{p}_{L,i} - p_L}^2\bigr] \leq \frac{N}{N-1}\,\mathbb{E}\!\bigl[\norm{\hat{p}_{L}^{\text{cent}} - p_L}^2\bigr] + \sigma_{\text{geo}}^2,
  \label{eq:decentral_vs_central}
\end{equation}
where $\sigma_{\text{geo}}^2$ accounts for the single-cable geometry limitation (one cable constrains a 1D subspace whereas $N$ cables can constrain 3D when in general position). For $N = 3$ cables in a non-collinear formation, the centralized estimator achieves full observability; each decentralized estimator relies on the prediction model to provide the missing constraint directions.