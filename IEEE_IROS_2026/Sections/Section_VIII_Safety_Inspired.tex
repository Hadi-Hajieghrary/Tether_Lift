While this work is primarily motivated by theoretical and algorithmic questions in decentralized geometric control, the design of the GPAC architecture is informed by established practices in safety-critical and assurance-oriented system engineering. 

This perspective is informed by established safety and assurance practices in autonomous systems and robotics, including functional safety lifecycles \cite{iso61508}, system-level safety analysis \cite{STPAleveson}, and operational risk frameworks for unmanned aircraft systems, such as the SORA and ASTM F38 standards \cite{aviationSafety}.

Rather than pursuing formal compliance, our objective is to demonstrate how safety-oriented design principles can inspire the development of robust decentralized control methods and contribute to their long-term deployability.

\subsection{Hazard-Oriented Decomposition of Cooperative Transport}

A central principle in assurance-driven system design is the early identification and mitigation of dominant hazards. In cooperative aerial transport, the primary hazards include:

\begin{enumerate}
    \item Loss of controllability due to cable slack or over-tension,
    \item Payload oscillations and resonant excitation,
    \item Reduced thrust margin caused by excessive vehicle tilt,
    \item Inter-agent collisions,
    \item Performance degradation under uncertainty and environmental disturbances.
\end{enumerate}

The GPAC architecture reflects this hazard-oriented perspective by decomposing control and estimation into layers that directly address these risks. Anti-swing regulation stabilizes pendular modes on $\mathbb{S}^2$, adaptive estimation mitigates load uncertainty, disturbance observers compensate unmodeled dynamics, and the CBF layer enforces physical constraints. This structure parallels risk allocation practices commonly used in safety engineering. 
Similar hazard-to-mitigation mappings are advocated in system-theoretic safety analysis and failure-oriented design methodologies \cite{STPAleveson, mlfmea}.

\subsection{Treatment of Uncertainty and Disturbances}

Robustness to uncertainty is a core requirement in safety-critical control systems. In cooperative transport, uncertainty arises from unknown payload mass, heterogeneous cable properties, and environmental disturbances.

GPAC addresses these uncertainties through decentralized concurrent learning and extended state observation. Each agent independently estimates its load share and rejects disturbances based solely on local measurements, thereby reducing dependence on centralized information. This approach mitigates common-cause failures arising from shared parameter estimates and enhances resilience against partial system failures.

From an assurance perspective, the combination of adaptive estimation and disturbance rejection serves as a systematic mechanism for maintaining performance under bounded uncertainty.
This treatment aligns with established fault-tolerant and learning-enabled system safety practices that emphasize online adaptation under uncertainty \cite{mlfmea,STPAleveson}.

\subsection{Runtime Enforcement of Operational Constraints}

Modern assurance frameworks emphasize runtime supervision to maintain operation within verified envelopes. In GPAC, this role is fulfilled by the Control Barrier Function safety filter, which enforces constraints on cable tension, swing rate, vehicle tilt, and inter-agent separation.

Unlike heuristic limiters, the CBF formulation provides mathematically grounded forward-invariance guarantees. Higher-order CBFs enable the treatment of relative-degree-two constraints, and the ISSf analysis bounds constraint violations under disturbances.
Control barrier functions have been widely adopted for runtime safety enforcement in safety-critical robotic systems \cite{ames2017control}.

Furthermore, the compatibility analysis in Theorem~5.4 ensures that safety interventions preserve the stability properties of the geometric attitude controller. This aligns with supervisory control concepts in flight-critical and robotic systems, where safety monitors must not compromise nominal closed-loop behavior.

\subsection{Hierarchical Architecture and Fault Containment}

Safety-oriented system architectures commonly employ hierarchical decomposition and time-scale separation to limit fault propagation. GPAC adopts a multi-rate cascade in which position control, attitude stabilization, adaptive estimation, and disturbance observation operate on distinct time scales.

This separation supports independent stability analyses and reduces coupling between estimation errors, control transients, and constraint enforcement. From a systems perspective, it facilitates modular verification and incremental validation.

In addition, the fully decentralized implementation avoids single points of failinherent to centralized coordination, thereby on, improving fault containment in multi-agent deployments.
Such architectural modularity is a recurring theme in assurance-oriented autonomy frameworks \cite{ul4600,koopman2017autonomous}.

\subsection{Failure Mode Coverage}

The architecture can be interpreted through a failure-mode-oriented lens, in which dominant risks are associated with explicit mitigation mechanisms.

\begin{table}[t]
\centering
\caption{Failure Modes and Corresponding Mitigation Mechanisms}
\label{tab:failure_modes}
\begin{tabular}{ll}
\hline
\textbf{Failure Mode} & \textbf{Mitigation Mechanism} \\
\hline
Payload mass uncertainty & Concurrent learning estimation \\
Environmental disturbances & ESO-based disturbance rejection \\
Cable slack or overload & CBF tension constraints \\
Payload oscillation & $\mathbb{S}^2$ anti-swing control \\
Excessive vehicle tilt & CBF tilt constraints \\
Inter-agent collision & CBF separation constraints \\
Sensor noise and dropout & ESKF fusion \\
\hline
\end{tabular}
\end{table}

This mapping reflects an implicit FMEA-style reasoning process that supports systematic analysis of system robustness.\cite{stamatis2003fmea}\cite{mlfmea}

\subsection{Implications for Research and Deployment}

Although GPAC is not intended as a certifiable control solution, its design reflects principles common to assured autonomous systems:

\begin{itemize}
    \item Explicit treatment of uncertainty and disturbances,
    \item Formal stability and boundedness guarantees,
    \item Runtime constraint enforcement,
    \item Modular, analyzable architecture,
    \item Absence of centralized failure points.
\end{itemize}

These properties suggest that geometric control methods, when integrated with adaptive estimation and supervisory safety layers, can form a foundation for future deployment-oriented cooperative aerial systems.
This perspective is consistent with recent calls for integrating formal methods, runtime monitoring, and system-level assurance in autonomous robotics research \cite{koopman2017autonomous, STPAleveson}.

More broadly, this work illustrates how assurance-oriented design concepts can inform the development of decentralized nonlinear control architectures without sacrificing theoretical rigor.