
% --- Geometric control literature ---

The geometric mechanics of cable-suspended transport have been studied extensively since the seminal work of Lee, Sreenath, and Kumar~\cite{lee2010geometric},\cite{sreenath2013geometric}, who formulated the coupled dynamics of a quadrotor--load system on the configuration manifold $\SEthree \times (\Sph^2)^N$ and derived controllers with provable almost-global asymptotic stability guarantees. In this framework, the attitude of each quadrotor~$i$ evolves on the special orthogonal group $\SOthree$, while the cable direction $q_i \in \Sph^2$ is a unit vector constrained to the two-sphere. The coupling between translational load dynamics and rotational cable swing is captured through the tangent-space projection $P(q) = I_{3\times 3} - qq^\top$~\eqref{eq:projection_s2}, and the attitude tracking error $e_R$~\eqref{eq:eR_model} is defined directly on $\SOthree$ via the vee map $(\cdot)^{\scriptscriptstyle\vee} : \sothree \to \R^3$, avoiding the singularities and unwinding phenomena inherent in Euler-angle or quaternion-based parameterizations. Subsequent extensions incorporated load swing suppression on $\Sph^2$ using the configuration error function $\Psi_q = 1 - q_d \cdot q$~\eqref{eq:psi_model} and the cable direction error $e_q = P(q)\,q_d$~\eqref{eq:eq_model}, together with geometric feedforward/feedback policies that yield exponential convergence of the cable direction to the desired equilibrium~\cite{lee2018geometric,goodarzi2014geometric}. These results constitute the theoretical gold standard for cable-suspended aerial manipulation.

% --- Limitations of centralized geometric approaches ---

However, many Lyapunov-certified geometric cooperative transport controllers assume centralized state knowledge. Each quadrotor's control law requires global knowledge of the total number of agents~$N$, the payload mass~$m_L$, the full system state---including the states of all other agents---and often the complete allocation of desired forces across the team. These assumptions permit elegant closed-form stability proofs but render the resulting controllers impractical for deployment. In realistic multi-agent scenarios, the team size~$N$ may change due to agent failures or additions, the payload mass~$m_L$ is typically uncertain, and reliable high-bandwidth inter-agent communication cannot be guaranteed. The gap between the mathematical elegance of geometric cooperative transport theory and the operational requirements of decentralized deployment remains largely unaddressed.

% --- Adaptive / decentralized literature ---

Conversely, the adaptive and decentralized multi-robot control literature addresses many of these practical constraints but typically at the cost of geometric fidelity. Consensus-based formation controllers~\cite{ren2005consensus}, distributed optimization methods, and adaptive impedance strategies have been proposed for cooperative manipulation tasks, but these approaches predominantly rely on linearized dynamics, Euclidean error metrics, and small-angle approximations that forfeit the global stability guarantees and singularity-free operation that geometric methods provide. In particular, linearized attitude representations break down during the large-angle maneuvers that commonly arise during payload pickup, aggressive trajectory tracking, and disturbance recovery---precisely the operating regimes where formal stability guarantees are most needed. Meanwhile, adaptive estimation techniques for uncertain parameters, such as model reference adaptive control (MRAC) and composite adaptation, have been developed for single-vehicle systems but have not been integrated with the coupled geometric dynamics of multi-agent cable-suspended transport. The concurrent learning paradigm~\cite{chowdhary2010concurrent,chowdhary2013exponentially}, which enables parameter convergence without persistent excitation (PE) by exploiting stored data, is particularly relevant for cooperative transport where the nominal hover condition provides limited excitation; yet existing applications remain confined to centralized single-vehicle settings.

% --- Safety-critical control gap ---

A parallel challenge concerns cable control safety-critical operation. Cable-suspended transport imposes multiple state constraints that must be enforced in real time: cables must remain taut to maintain controllability, cable angles must stay within bounds to prevent entanglement and load instability, swing rates must be limited to avoid resonant excitation, and inter-agent collisions must be prevented. Control Barrier Functions (CBFs)~\cite{ames2017control,ames2019control} have emerged as a principled framework for enforcing such constraints via online quadratic program (QP) filters that minimally modify a nominal controller to guarantee forward invariance of a safe set. However, integrating CBF-based safety filters with geometric controllers on nonlinear manifolds---while preserving the underlying Lyapunov stability certificates---remains challenging, particularly in the decentralized multi-agent setting where coupled constraints and attitude stability certificates must coexist.

% --- Flexible cable modeling gap ---

Furthermore, the majority of existing geometric transport controllers assume idealized cable models---either rigid links or massless inextensible strings---that cannot capture the rich dynamics of real flexible cables. In practice, cables exhibit compliance, wave propagation, slack-to-taut transitions, and distributed inertia effects that significantly influence the coupled system behavior. These phenomena are especially pronounced during the critical payload pickup phase, where ropes transition from slack to taut and impulsive forces can destabilize the formation. Bead-chain discretizations, in which the cable is modeled as a series of point masses connected by tension-only spring-damper elements, provide a physically faithful representation of these effects~\cite{williams2009dynamics}, but their integration with geometric controllers has received limited attention.

% --- Sensor realism gap ---

Finally, the vast majority of geometric control results for cooperative transport are validated under the assumption of perfect state feedback. In practice, onboard state estimation must fuse noisy, heterogeneous sensor measurements---inertial measurement units (IMUs) subject to Gauss-Markov bias drift, GPS receivers with intermittent dropouts, and barometric altimeters with correlated noise---through nonlinear filtering pipelines such as the Error-State Kalman Filter (ESKF)~\cite{sola2017quaternion}. Wind disturbances, modeled here via the Dryden turbulence spectrum~\cite{moorhouse1982us}, introduce additional unmodeled forces that must be rejected. The robustness of geometric cooperative transport controllers to realistic estimation errors and environmental disturbances remains largely uncharacterized.

% ============================================================
% Thesis statement
% ============================================================

This paper presents a unified framework that bridges these two bodies of literature---geometric control theory and decentralized adaptive systems---in a unified framework for cooperative cable-suspended aerial transport. We propose the \textbf{Geometric Position and Attitude Control (GPAC)} architecture: a four-layer hierarchical controller in which each quadrotor operates with \emph{zero knowledge} of the total agent count~$N$, the payload mass~$m_L$, or any other agent's state, yet the closed-loop system maintains geometric control on the full $\SEthree \times \Sph^2$ configuration manifold with Lyapunov-certifiable convergence guarantees. The architecture is validated in a high-fidelity Drake-based simulation environment incorporating flexible bead-chain cable dynamics, realistic onboard sensors, and Dryden wind turbulence.
