Cooperative aerial transport—in which multiple unmanned aerial vehicles (UAVs) jointly manipulate a common payload via cable suspensions—has emerged as a compelling paradigm for extending payload capacity, workspace, and fault tolerance beyond the limits of any single platform. Applications range from construction logistics and emergency supply delivery to the assembly of large-scale structures in environments inaccessible to ground vehicles. In addition to performance, such systems are inherently safety-critical: failures in thrust generation, cable integrity, state estimation, coordination, or environmental disturbance rejection can rapidly lead to loss of controllability, payload instability, or inter-agent collisions. Ensuring robust operation in the presence of these hazards is therefore central to practical deployment.

The fundamental appeal is clear: $N$ quadcopters, each rated for a maximum thrust $T_i^{\max}$, can collectively transport payloads approaching $N$ times the capacity of an individual agent, while the spatial distribution of attachment points affords controllability over the payload's full six-degree-of-freedom pose. 
Realizing this potential in practice requires controllers that are simultaneously geometrically rigorous (respecting the nonlinear manifold structure of the configuration space), operationally decentralized, and capable of systematically mitigating dominant failure modes, without reliance on centralized coordination or high-bandwidth communication.
