Cooperative aerial transport---in which multiple unmanned aerial vehicles (UAVs) jointly manipulate a common payload via cable suspensions---has emerged as a compelling paradigm for extending the payload capacity, workspace, and fault tolerance of rotorcraft beyond the limits of any single platform. Applications range from construction logistics and emergency supply delivery to the assembly of large-scale structures in environments inaccessible to ground vehicles. The fundamental appeal is clear: $N$ quadcopters, each rated for a maximum thrust $T_i^{\max}$, can collectively transport payloads approaching $N$ times the capacity of an individual agent, while the spatial distribution of attachment points affords controllability over the payload's full six-degree-of-freedom pose. Yet realizing this potential in practice demands controllers that are simultaneously \emph{geometrically rigorous}---respecting the nonlinear manifold structure of the configuration space---and \emph{operationally decentralized}, requiring neither a central coordinator nor inter-agent communication during flight.

% --- Geometric control literature ---

The geometric mechanics of cable-suspended transport have been studied extensively since the seminal work of Lee, Sreenath, and Kumar~\cite{lee2010geometric,sreenath2013geometric}, who formulated the coupled dynamics of a quadrotor--load system on the configuration manifold $\SEthree \times (\Sph^2)^N$ and derived controllers with provable almost-global asymptotic stability guarantees. In this framework, the attitude of each quadrotor~$i$ evolves on the special orthogonal group $\SOthree$, while the cable direction $q_i \in \Sph^2$ is a unit vector constrained to the two-sphere. The coupling between translational load dynamics and rotational cable swing is captured through the tangent-space projection $P(q) = I_{3\times 3} - qq^\top$~\eqref{eq:projection_s2}, and the attitude tracking error $e_R$~\eqref{eq:eR_model} is defined directly on $\SOthree$ via the vee map $(\cdot)^{\scriptscriptstyle\vee} : \sothree \to \R^3$, avoiding the singularities and unwinding phenomena inherent in Euler-angle or quaternion-based parameterizations. Subsequent extensions incorporated load swing suppression on $\Sph^2$ using the configuration error function $\Psi_q = 1 - q_d \cdot q$~\eqref{eq:psi_model} and the cable direction error $e_q = P(q)\,q_d$~\eqref{eq:eq_model}, together with geometric feedforward/feedback policies that yield exponential convergence of the cable direction to the desired equilibrium~\cite{lee2018geometric,goodarzi2014geometric}. These results constitute the theoretical gold standard for cable-suspended aerial manipulation.

% --- Limitations of centralized geometric approaches ---

However, many Lyapunov-certified geometric cooperative transport controllers assume centralized state knowledge. Each quadrotor's control law requires global knowledge of the total number of agents~$N$, the payload mass~$m_L$, the full system state---including the states of all other agents---and often the complete allocation of desired forces across the team. These assumptions permit elegant closed-form stability proofs but render the resulting controllers impractical for deployment. In realistic multi-agent scenarios, the team size~$N$ may change due to agent failures or additions, the payload mass~$m_L$ is typically uncertain, and reliable high-bandwidth inter-agent communication cannot be guaranteed. The gap between the mathematical elegance of geometric cooperative transport theory and the operational requirements of decentralized deployment remains largely unaddressed.

% --- Adaptive / decentralized literature ---

Conversely, the adaptive and decentralized multi-robot control literature addresses many of these practical constraints but typically at the cost of geometric fidelity. Consensus-based formation controllers~\cite{ren2005consensus}, distributed optimization methods, and adaptive impedance strategies have been proposed for cooperative manipulation tasks, but these approaches predominantly rely on linearized dynamics, Euclidean error metrics, and small-angle approximations that forfeit the global stability guarantees and singularity-free operation that geometric methods provide. In particular, linearized attitude representations break down during the large-angle maneuvers that commonly arise during payload pickup, aggressive trajectory tracking, and disturbance recovery---precisely the operating regimes where formal stability guarantees are most needed. Meanwhile, adaptive estimation techniques for uncertain parameters, such as model reference adaptive control (MRAC) and composite adaptation, have been developed for single-vehicle systems but have not been integrated with the coupled geometric dynamics of multi-agent cable-suspended transport. The concurrent learning paradigm~\cite{chowdhary2010concurrent,chowdhary2013exponentially}, which enables parameter convergence without persistent excitation (PE) by exploiting stored data, is particularly relevant for cooperative transport where the nominal hover condition provides limited excitation; yet existing applications remain confined to centralized single-vehicle settings.

% --- Safety-critical control gap ---

A parallel challenge concerns safety-critical operation. Cable-suspended transport imposes multiple state constraints that must be enforced in real time: cables must remain taut to maintain controllability, cable angles must stay within bounds to prevent entanglement and load instability, swing rates must be limited to avoid resonant excitation, and inter-agent collisions must be prevented. Control Barrier Functions (CBFs)~\cite{ames2017control,ames2019control} have emerged as a principled framework for enforcing such constraints via online quadratic program (QP) filters that minimally modify a nominal controller to guarantee forward invariance of a safe set. However, integrating CBF-based safety filters with geometric controllers on nonlinear manifolds---while preserving the underlying Lyapunov stability certificates---remains challenging, particularly in the decentralized multi-agent setting where coupled constraints and attitude stability certificates must coexist.

% --- Flexible cable modeling gap ---

Furthermore, the majority of existing geometric transport analyses assume idealized cable models---either rigid links or massless inextensible strings---that cannot capture the rich dynamics of real flexible cables. In practice, cables exhibit compliance, wave propagation, slack-to-taut transitions, and distributed inertia effects that significantly influence the coupled system behavior. These phenomena are especially pronounced during the critical payload pickup phase, where ropes transition from slack to taut and impulsive forces can destabilize the formation. Bead-chain discretizations, in which the cable is modeled as a series of point masses connected by tension-only spring-damper elements, provide a physically faithful representation of these effects~\cite{williams2009dynamics}, but their integration with geometric controllers has received limited attention.

% --- Sensor realism gap ---

Finally, the vast majority of geometric control results for cooperative transport are validated under the assumption of perfect state feedback. In practice, onboard state estimation must fuse noisy, heterogeneous sensor measurements---inertial measurement units (IMUs) subject to Gauss-Markov bias drift, GPS receivers with intermittent dropouts, and barometric altimeters with correlated noise---through nonlinear filtering pipelines such as the Error-State Kalman Filter (ESKF)~\cite{sola2017quaternion}. Wind disturbances, modeled here via the Dryden turbulence spectrum~\cite{moorhouse1982us}, introduce additional unmodeled forces that must be rejected. The robustness of geometric cooperative transport controllers to realistic estimation errors and environmental disturbances remains largely uncharacterized.

% ============================================================
% Thesis statement
% ============================================================

This paper presents a unified framework that bridges these two bodies of literature---geometric control theory and decentralized adaptive systems---in a unified framework for cooperative cable-suspended aerial transport. We propose the \textbf{Geometric Position and Attitude Control (GPAC)} architecture: a four-layer hierarchical controller in which each quadrotor operates with \emph{zero knowledge} of the total agent count~$N$, the payload mass~$m_L$, or any other agent's state, yet the closed-loop system maintains geometric control on the full $\SEthree \times \Sph^2$ configuration manifold with Lyapunov-certifiable convergence guarantees. The architecture is validated in a high-fidelity Drake-based simulation environment incorporating flexible bead-chain cable dynamics, realistic onboard sensors, and Dryden wind turbulence.

% ============================================================
% Specific contributions
% ============================================================

The specific contributions of this work are as follows. At the highest level, the GPAC architecture decomposes the cooperative transport problem into $N$ identical single-agent subproblems. Each drone independently estimates only its own share of the payload mass, $\hat{\theta}_i \approx m_L/N$, using local cable tension and direction measurements. The key insight is that when all agents do this simultaneously, their combined forces automatically sum to the correct total---no explicit coordination required. A modular safety filter overlays the resulting controller to enforce physical constraints (cable tautness, collision avoidance) with minimal performance impact.

\begin{enumerate}
  \item \textbf{Decentralized geometric cooperative transport with formal stability guarantees.}
  We derive an operationally decentralized control law---no peer-to-peer state exchange is required at runtime; each agent receives only the shared reference trajectory via a common broadcast---in which each quadrotor's policy depends only on its own state and local cable-tension and direction measurements. No drone requires knowledge of~$N$ or~$m_L$. The control is formulated directly on $\SOthree \times \Sph^2$ using the attitude error~\eqref{eq:eR_model} and the $\Sph^2$ cable direction error~\eqref{eq:eq_model}, preserving global geometric structure. Unlike prior decentralized controllers that linearize around hover or use Euclidean error metrics, this is, to our knowledge, the first decentralized geometric controller for multi-UAV cable transport that operates directly on the nonlinear $\mathrm{SE}(3) \times (\mathbb{S}^2)^N$ manifold. The key theoretical novelty is proving that independent local estimation of $\hat{\theta}_i$ yields implicit coordination~\eqref{eq:force_convergence} without requiring any communication or parameter sharing---a property that does not hold for linearized or impedance-based decentralized approaches. We show that when each agent independently estimates its load share parameter $\hat{\theta}_i \to m_L/N$, the summed forces automatically converge to the correct total without explicit coordination:
  \begin{equation}
    \sum_{i=1}^{N} F_i = \sum_{i=1}^{N} \hat{\theta}_i \cdot u \;\longrightarrow\; m_L \cdot u.
    \label{eq:force_convergence}
  \end{equation}

  \item \textbf{Concurrent learning adaptive estimation without persistent excitation.}
  We introduce a decentralized adaptive estimation scheme in which each drone independently estimates the ratio $\hat{\theta} = m_L / N$ using only its local cable tension $ T_i$, cable angle $ \ phi_i$, and load acceleration estimate. The key insight is that
  \begin{equation}
    \frac{T_i \cos\phi_i}{\norm{g\,e_3 + a_L}} \;\longrightarrow\; \frac{m_L}{N},
    \label{eq:theta_estimation}
  \end{equation}
  the quantity each drone actually needs, without requiring knowledge of either~$m_L$ or~$N$ individually. A concurrent learning algorithm~\cite{chowdhary2010concurrent} with a rank-maximizing history stack of regressor--output pairs $(Y_j, z_j)$ ensures parameter convergence via the update law
  \begin{equation}
    \dot{\hat{\theta}} = \Gamma\!\left(Y^\top s + \rho \sum_{j=1}^{M} Y_j^\top\!\bigl(Y_j \hat{\theta} - z_j\bigr)\right)\!,
    \label{eq:concurrent_learning}
  \end{equation}
  without the persistent excitation condition---which is critical because cooperative hover, the nominal operating condition, provides insufficient excitation for classical adaptive laws. The theoretical contribution is the adaptation of concurrent learning to a decentralized multi-agent geometric setting, where each agent's regressor is constructed from purely local cable measurements. This is non-trivial because the regressor vector depends on the coupled system dynamics, yet we show that the local tension equilibrium~\eqref{eq:theta_estimation} provides a sufficient scalar parametric model for each agent independently.

  \item \textbf{Multi-rate hierarchical architecture with time-scale separation.}
  The GPAC architecture implements a four-layer cascade with deliberate bandwidth separation:
  \begin{itemize}
    \item \emph{Layer 1 (${\sim}$50\,Hz effective):} Position tracking with $\Sph^2$ anti-swing control, producing a desired thrust direction from the force command
    \begin{equation}
      F_{\mathrm{des}} = -K_p(p - p_d) - K_d(\dot{p} - \dot{p}_d) + \hat{\theta}\bigl(g\,e_3 + \ddot{p}_d^L\bigr) + F_{\mathrm{swing}},
      \label{eq:position_control}
    \end{equation}
    where $F_{\mathrm{swing}} = k_q\,e_q + k_\omega(q_d \times \omega_q)$ damps cable oscillations on the tangent space $T_q\Sph^2$.
    \item \emph{Layer 2 (200\,Hz):} Geometric $\SOthree$ attitude tracking with the control torque
    \begin{equation}
      \tau = -K_R\,e_R - K_\Omega\,e_\Omega + \Omega \times J\Omega + J\!\bigl(\hatmap{\Omega}\,R^\top R_d\,\Omega_d - R^\top R_d\,\dot{\Omega}_d\bigr) - \hat{d},
      \label{eq:attitude_control}
    \end{equation}
    where $e_\Omega = \Omega - R^\top R_d\,\Omega_d$ is the angular velocity error and $\hat{d}$ is the ESO disturbance estimate.
    \item \emph{Layer 3 (50\,Hz):} Concurrent learning parameter estimation via~\eqref{eq:concurrent_learning}.
    \item \emph{Layer 4 (continuous, $\omega_0 = 50$\,rad/s):} Third-order Extended State Observer (ESO) per translational axis,
    \begin{equation}
      \dot{\hat{x}}_1 = \hat{x}_2 + 3\omega_0\tilde{x}, \;\;
      \dot{\hat{x}}_2 = \hat{x}_3 + 3\omega_0^2\tilde{x}, \;\;
      \dot{\hat{x}}_3 = \omega_0^3\tilde{x},
      \label{eq:eso}
    \end{equation}
    with $\tilde{x} = x - \hat{x}_1$ and bandwidth $\omega_0 = 50$\,rad/s, estimating the lumped disturbance $\hat{d} = \hat{x}_3$.
  \end{itemize}
  The deliberate time-scale separation enables independent Lyapunov analysis of each layer while the cascade composition preserves overall closed-loop stability, making the decentralized stability proof tractable via singular perturbation arguments.

  \item \textbf{Modular CBF safety filter preserving geometric stability certificates.}
  We design a Control Barrier Function safety layer implemented as an online QP that bounds safety constraint violations via ISSf margins. Defining barrier functions for cable tautness $h_1 = \norm{p_q - p_L} - L_{\min}$, cable angle $h_2 = \cos\theta - \cos\theta_{\max}$, swing rate bounds, vehicle tilt limits, and inter-agent collision avoidance, the safety filter solves
  \begin{equation}
    \min_{u,\,\delta} \;\norm{u - u_{\mathrm{nom}}}^2 + \lambda\,\delta^2 \quad
    \text{s.t.} \;\; \dot{h}_k + \alpha_k h_k \geq -\delta_k, \;\; \forall\, k,
    \label{eq:cbf_qp_intro}
  \end{equation}
  where $u_{\mathrm{nom}}$ is the geometric controller output, $\alpha_k > 0$ are class-$\mathcal{K}$ coefficients, and the slack variables~$\delta_k$ resolve potential conflicts between competing constraints and the geometric controller's convergence requirements. Higher-order CBF (HOCBF) formulations handle relative-degree-two constraints, such as cable tautness, and a second-order Butterworth filter provides smooth tension-rate estimates for the constraint Jacobians. The filter operates as a modular overlay, minimally modifying the nominal geometric control input to enforce constraint satisfaction. The key theoretical result (Theorem~\ref{thm:compatibility}) establishes that the CBF safety filter preserves the almost-global exponential stability certificate of the geometric attitude controller---a compatibility proof that requires bounding the attitude perturbation induced by the force modification on $\SOthree$. This compatibility argument, specific to the cable-transport setting, shows that the CBF-induced attitude perturbation remains within the almost-global stability region of the geometric controller under bounded disturbances and within the ISSf inflation $\mathcal{C}_\mu$.

  \item \textbf{Sensor-realistic validation with flexible cable dynamics.}
  The complete framework is validated in a Drake-based~\cite{drake2024} multibody simulation incorporating:
  \begin{enumerate}
    \item Flexible cables modeled as bead-chain discretizations with $n_b = 8$ point-mass beads per cable and $n_b + 1 = 9$ tension-only spring-damper segments that naturally capture slack-to-taut transitions, wave propagation, and distributed inertia;
    \item A full onboard sensor suite comprising IMUs with Gauss--Markov bias dynamics (noise density $\sigma_a = 0.004$\,m/s$^2$/$\sqrt{\text{Hz}}$, $\sigma_g = 5 \times 10^{-4}$\,rad/s/$\sqrt{\text{Hz}}$, bias time constant $\tau_b = 3600$\,s), GPS receivers with stochastic dropouts ($\sigma_{xy} = 0.02$\,m, $\sigma_z = 0.05$\,m), and barometric altimeters with correlated noise ($\sigma_w = 0.3$\,m, correlation time $\tau_c = 5$\,s, drift rate $0.002$\,m/s);
    \item A 15-state ESKF with error state $\delta x = [\delta p,\; \delta v,\; \delta\theta,\; \delta b_a,\; \delta b_g]^\top \in \R^{15}$ fusing these heterogeneous measurements;
    \item Dryden-spectrum wind turbulence with turbulence intensities $\sigma_u = \sigma_v = 0.5$\,m/s, $\sigma_w = 0.25$\,m/s, and altitude-dependent scaling.
  \end{enumerate}
  This constitutes a high-fidelity simulation study incorporating flexible cable dynamics, multi-rate sensor models, and atmospheric turbulence, providing a rigorous validation environment for the proposed architecture.
\end{enumerate}

% ============================================================
% Key insight recap
% ============================================================

Taken together, these contributions demonstrate that it is possible to simultaneously achieve operational decentralization, geometric rigor on the proper configuration manifold, ISSf safety guarantees bounding constraint violations under disturbances, and robustness to realistic sensing and environmental conditions---addressing the critical deployment gap that has prevented geometric cooperative transport theory from transitioning to operational multi-agent systems. The key enabling insight is the $\hat{\theta} = m_L/N$ estimation architecture, which reduces the cooperative transport problem to~$N$ identical single-agent problems whose solutions automatically compose to the correct collective behavior without explicit coordination or parameter sharing.

% ============================================================
% Paper organization
% ============================================================

The remainder of this paper is organized as follows. Section~\ref{sec:modeling} formalizes the multi-body dynamic model on $\SEthree \times (\Sph^2)^N$, including the bead-chain cable model, sensor noise models, and wind disturbance characterization. Section~\ref{sec:control} presents the GPAC hierarchical control architecture, deriving the control laws for each layer and establishing the individual and cascade stability properties. Section~\ref{sec:estimation} details the decentralized adaptive estimation scheme and its convergence guarantees under concurrent learning. Section~\ref{sec:safety} formulates the CBF safety filter and proves compatibility with the geometric controller. Section~\ref{sec:simulation} describes the Drake simulation environment, sensor models, and implementation details. Section~\ref{sec:results} presents comprehensive simulation results for multi-phase transport scenarios including hover, ascent, lateral translation, and descent under wind disturbance and sensor noise, with comparisons to centralized baselines and ablation studies. Section~\ref{sec:conclusion} concludes with a discussion of limitations and directions for future work, including experimental validation on physical platforms.
