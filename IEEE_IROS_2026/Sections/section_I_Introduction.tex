% !TEX root = ../Main.tex
Cooperative aerial transport involves several unmanned aerial vehicles (UAVs) working together to carry a payload with cables. This approach offers greater capacity, a larger workspace, and better fault tolerance than using a single UAV. However, these systems are safety-critical. Issues such as cable slack, excessive swing, vehicle tilt, collisions between UAVs, and disturbances can all cause loss of control or make the payload unstable. To achieve cooperation among $N$ quadrotors, controllers must handle the nonlinear system dynamics, operate without centralized coordination, and address the main failure modes.

Lee, Sreenath, and Kumar~\cite{lee2010geometric, sreenath2013geometric} developed a method for cable-suspended transport using the full nonlinear configuration manifold, achieving almost global stability without problems from Euler angle singularities. Later, they added anti-swing control for the cables~\cite{lee2018geometric}. Sharma and Sundaram~\cite{sharma2023geometric} created a geometric controller for multi-UAV payload transfer that does not need link information. Sun et al.~\cite{sun2025agile} showed agile cooperative cable manipulation with online kinodynamic planning. However, these controllers need centralized state information or full feedback about the payload. Each quadrotor must know the number of cooperating UAVs~$N$ and the payload mass~$m_L$, or the states of all other UAVs. This is not practical if communication is unreliable.

Consensus-based formation controllers and distributed optimization methods use linearized dynamics and Euclidean error measures. These methods lose global stability during large-angle maneuvers, which is when stability is most important. Wang et al.~\cite{wang2024automultilift} introduced Auto-Multilift, a distributed learning system that tunes model predictive control (MPC) cost functions online for cooperative load transport. However, this method relies on optimization-based MPC instead of geometric control and does not offer Lyapunov stability guarantees. Concurrent learning~\cite{chowdhary2010concurrent, chowdhary2013exponentially} allows convergence without persistent excitation by using stored data, which matters because cooperative hover does not provide enough excitation. So far, concurrent learning has only been used in centralized, single-vehicle cases.

Cable-suspended transport requires real-time state constraints. The cables must stay taut, cable angles must be within limits, swing rates must be controlled, and collisions must be avoided. Control Barrier Functions (CBFs)~\cite{ames2017control, ames2019control} enforce these constraints using online quadratic programs (QPs) that make minimal changes to the main controller. Yang and Xie~\cite{yang2025robust} combined CBFs with disturbance estimators to improve safety for single-quadrotor cable-suspended payloads. They found that disturbance-observer CBFs are less conservative when the model is uncertain. However, combining CBFs with geometric controllers on nonlinear manifolds while maintaining Lyapunov guarantees remains an open problem. This is even harder in decentralized multi-agent systems, where constraints and attitude stability must be managed simultaneously.

Most studies on geometric transport assume cables are rigid and state feedback is perfect. In reality, cables are flexible, show wave effects, and can go from slack to taut. These factors have a big impact on system behavior~\cite{williams2009dynamics}. Onboard estimation must combine noisy IMU and GPS data with nonlinear filters~\cite{sola2017quaternion}. Wind also adds unpredictable forces. How well geometric cooperative controllers handle these real-world issues remains poorly understood. In short, there is no current framework that provides geometric manifold control, decentralized adaptive estimation without persistent excitation, and real-time safety certification for multi-UAV cable transport.

To close these gaps, safety-critical engineering focuses on identifying and reducing major hazards early~\cite{STPAleveson}. Rather than building a single large controller and checking safety afterward, each GPAC layer targets a specific transport risk: anti-swing control stabilizes swinging, adaptive estimation handles load uncertainty, the extended state observer (ESO) handles disturbances, and the CBF layer acts as a safety supervisor during operation. The system uses a layered, multi-rate structure with separate timescales to limit fault spread, and its fully decentralized design removes single points of failure~\cite{koopman2017autonomous}. Table~\ref{tab:failure_modes} in Section~\ref{sec:results} shows this mapping with real data.

Building on this idea, the GPAC architecture breaks down the cooperative transport problem into $N$ identical single-agent tasks. The key insight is that each quadrotor can estimate its share of the payload mass, $\hat{\theta}_i \approx m_L/N$, using only its own cable measurements. The combined forces from all agents then add up to the correct total:
\begin{equation}
  \sum_{i=1}^{N} F_i = \sum_{i=1}^{N} \hat{\theta}_i \cdot u \;\longrightarrow\; m_L \cdot u,
  \label{eq:force_convergence}
\end{equation}
In this setup, $u := g\,e_3 + \ddot{p}_L^d$ is the common payload acceleration demand, and every agent calculates it the same way from the shared reference. This approach means there is no need for a central coordinator, for exchanging payload or cable states, for consensus protocols, or for knowing the payload mass in advance. The main contributions are:
\begin{enumerate}[leftmargin=*, itemsep=2pt]
  \item The decentralized controller for multi-UAV cable transport operates directly on the nonlinear manifold without linearization, requiring no payload state, cable state, or adaptive-parameter exchange.

  \item Each agent estimates $\hat{\theta}_i \to m_L/N$ from local cable tension and direction with exponential convergence---even during near-hover conditions where classical adaptive laws fail.

  \item A modular safety layer enforcing cable tautness, angle, tilt, swing rate, and collision constraints with input-to-state safety (ISSf) guarantees. Theorem~\ref{thm:compatibility} proves the CBF-induced modifications preserve almost-global exponential attitude stability on $\SOthree$.

  \item High-fidelity Drake-based simulation with bead-chain cables, onboard sensor fusion, and Dryden wind turbulence, achieving $23.5 \pm 1.5$\,cm payload RMSE at sub-MFLOP per-agent cost.
\end{enumerate}
