% !TEX root = ../Main.tex
Cooperative aerial transport uses multiple unmanned aerial vehicles (UAVs) to jointly carry a payload suspended by cables. This increases capacity, workspace, and fault tolerance compared to single platforms. These systems are inherently safety-critical. Cable slack, excessive swing, vehicle tilt, inter-agent collisions, and disturbances can each result in loss of controllability or payload instability. Achieving $N$-quadrotor cooperation requires controllers that respect the nonlinear manifold structure, function without centralized coordination, and systematically address dominant failure modes.

Lee, Sreenath, and Kumar~\cite{lee2010geometric, sreenath2013geometric} formulated cable-suspended transport on the full nonlinear configuration manifold. They achieved almost-global stability without Euler-angle singularities. Later work~\cite{lee2018geometric} added anti-swing regulation of pendular cable dynamics. Sharma and Sundaram~\cite{sharma2023geometric} introduced a geometric controller for multi-UAV payload transfer that does not require link information. Sun et al.~\cite{sun2025agile} demonstrated agile cooperative cable manipulation using online kinodynamic planning. However, these controllers require centralized state knowledge or full-state payload feedback. Each quadrotor must know the number of cooperating agents~$N$ and the payload mass~$m_L$, or all peers' states. This is impractical when communication is unreliable.

Consensus-based formation controllers and distributed optimization approaches use linearized dynamics and Euclidean error metrics. These forfeit global stability during large-angle maneuvers, precisely when guarantees are most critical. Wang et al.~\cite{wang2024automultilift} introduced Auto-Multilift, a distributed learning framework. It tunes model predictive control (MPC) cost functions online for cooperative load transport. However, this approach depends on optimization-based MPC, not geometric control. It does not provide Lyapunov stability certificates. Concurrent learning~\cite{chowdhary2010concurrent, chowdhary2013exponentially} enables convergence without persistent excitation (PE) by utilizing stored data. This is relevant because cooperative hover provides insufficient excitation. Existing concurrent learning applications are restricted to centralized, single-vehicle scenarios.

Cable-suspended transport imposes real-time state constraints. Cables must remain taut, cable angles must be bounded, swing rates must be limited, and collisions must be prevented. Control Barrier Functions (CBFs)~\cite{ames2017control, ames2019control} enforce these constraints via online quadratic programs (QPs) that minimally modify a nominal controller. Yang and Xie~\cite{yang2025robust} combined CBFs with disturbance estimators for single-quadrotor cable-suspended payload safety. They showed that disturbance-observer CBFs reduce conservatism under model uncertainty. However, integrating CBFs with geometric controllers on nonlinear manifolds while preserving Lyapunov certificates remains an open problem. This challenge becomes even more intense in decentralized multi-agent settings, where coupled constraints and attitude stability must be maintained simultaneously.

Most geometric transport studies assume rigid cables and perfect state feedback. In practice, real cables exhibit compliance, wave propagation, and slack-to-taut transitions. These effects significantly influence system behavior~\cite{williams2009dynamics}. Onboard estimation must combine noisy IMU and GPS data using nonlinear filters~\cite{sola2017quaternion}. Wind introduces unmodeled forces. The robustness of geometric cooperative controllers under these conditions remains largely uncharacterized. In summary, no existing framework simultaneously achieves geometric manifold control, decentralized adaptive estimation without persistent excitation, and runtime safety certification for multi-UAV cable transport.

To address these gaps, a central principle in safety-critical engineering is the early identification and systematic mitigation of dominant hazards~\cite{STPAleveson}. Instead of designing a monolithic controller and verifying safety after development, each GPAC layer addresses a specific transport hazard: anti-swing regulation stabilizes pendular modes, adaptive estimation mitigates load uncertainty, the extended state observer (ESO) compensates for disturbances, and the CBF layer enforces constraints as a runtime supervisor. The hierarchical multi-rate structure with timescale separation limits fault propagation, and the fully decentralized implementation eliminates single points of failure~\cite{koopman2017autonomous}. Table~\ref{tab:failure_modes} in Section~\ref{sec:results} validates this mapping with measured data.

Expanding on this principle, the GPAC architecture decomposes the cooperative transport problem into $N$ identical single-agent subproblems based on a key insight: each quadrotor independently estimates its payload mass share $\hat{\theta}_i \approx m_L/N$ from local cable measurements, and the combined forces automatically sum to the correct total:
\begin{equation}
  \sum_{i=1}^{N} F_i = \sum_{i=1}^{N} \hat{\theta}_i \cdot u \;\longrightarrow\; m_L \cdot u,
  \label{eq:force_convergence}
\end{equation}
In this setup, $u := g\,e_3 + \ddot{p}_L^d$ represents the common payload acceleration demand, which is computed identically by every agent from the shared reference. This implicit coordination removes the need for a centralized coordinator, payload or cable state exchange, consensus protocols, or prior knowledge of the payload mass. The specific contributions are as follows:
\begin{enumerate}[leftmargin=*, itemsep=2pt]
  \item The decentralized controller for multi-UAV cable transport operates directly on the nonlinear manifold without linearization, requiring no payload state, cable state, or adaptive-parameter exchange.

  \item Each agent estimates $\hat{\theta}_i \to m_L/N$ from local cable tension and direction with exponential convergence---even during near-hover conditions where classical adaptive laws fail.

  \item A modular safety layer enforcing cable tautness, angle, tilt, swing rate, and collision constraints with input-to-state safety (ISSf) guarantees. Theorem~\ref{thm:compatibility} proves the CBF-induced modifications preserve almost-global exponential attitude stability on $\SOthree$.

  \item High-fidelity Drake-based simulation with bead-chain cables, onboard sensor fusion, and Dryden wind turbulence, achieving $23.5 \pm 1.5$\,cm payload RMSE at sub-MFLOP per-agent cost.
\end{enumerate}
