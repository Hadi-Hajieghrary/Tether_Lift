% !TEX root = ../Main.tex
Cooperative aerial transport---in which multiple unmanned aerial vehicles (UAVs) jointly manipulate a common payload via cable suspensions---extends payload capacity, workspace, and fault tolerance beyond the limits of any single platform. Applications span construction logistics, emergency supply delivery, and assembly of large-scale structures in environments inaccessible to ground vehicles. Such systems are inherently safety-critical: cable slack, excessive swing, vehicle tilt, inter-agent collisions, and environmental disturbances can each independently cause loss of controllability or payload instability. Realizing the potential of $N$ cooperative quadrotors therefore demands controllers that simultaneously respect the nonlinear manifold structure of the configuration space, operate without centralized coordination, and systematically mitigate these dominant failure modes.

\subsection{Related Work and Open Gaps}

\textit{Geometric control.}
Lee, Sreenath, and Kumar~\cite{lee2010geometric, sreenath2013geometric} formulated cable-suspended transport on $\SEthree \times (\Sph^2)^N$, with attitude on $\SOthree$ and cable direction $q_i \in \Sph^2$, yielding almost-global stability guarantees free of Euler-angle singularities. Extensions~\cite{lee2018geometric, goodarzi2014geometric} added anti-swing control via $\Psi_q = 1 - q_d \cdot q$. Sharma and Sundaram~\cite{sharma2023geometric} proposed a tractable geometric controller for multi-UAV payload transfer that does not require link information, and Sun et al.~\cite{sun2025agile} recently demonstrated agile cooperative cable manipulation with online kinodynamic planning. However, these controllers assume centralized state knowledge or require full-state payload feedback: every drone must know $N$, $m_L$, or all peers' states---impractical when team size may change or communication cannot be guaranteed. A comprehensive survey of cable-suspended aerial transport is provided in~\cite{estevez2024review}.

\textit{Decentralized and adaptive methods.}
Consensus-based formation controllers~\cite{ren2005consensus} and distributed optimization rely on linearized dynamics and Euclidean error metrics that forfeit global stability during large-angle maneuvers---precisely when guarantees are most needed. Wang et al.~\cite{wang2024automultilift} proposed Auto-Multilift, a distributed learning framework that tunes MPC cost functions online for cooperative load transport, but it relies on optimization-based MPC rather than geometric control and does not provide Lyapunov stability certificates. The concurrent learning paradigm~\cite{chowdhary2010concurrent, chowdhary2013exponentially}, enabling convergence without persistent excitation (PE) by exploiting stored data, is particularly relevant because cooperative hover provides insufficient excitation for classical adaptive laws; yet existing applications remain confined to centralized single-vehicle settings.

\textit{Safety-critical control.}
Cable-suspended transport imposes real-time state constraints: cables must remain taut, cable angles bounded, swing rates limited, and collisions prevented. Control Barrier Functions (CBFs)~\cite{ames2017control, ames2019control} enforce such constraints via online QPs that minimally modify a nominal controller. Yang and Xie~\cite{yang2025robust} recently combined CBFs with disturbance estimators for single-quadrotor cable-suspended payload safety, demonstrating that disturbance-observer CBFs reduce conservatism under model uncertainty. However, integrating CBFs with geometric controllers on nonlinear manifolds while preserving Lyapunov certificates remains open, particularly in decentralized multi-agent settings where coupled constraints and attitude stability certificates must coexist.

\textit{Simulation realism.}
Most geometric transport results assume rigid cables and perfect state feedback. Real cables exhibit compliance, wave propagation, and slack-to-taut transitions that significantly affect system behavior~\cite{williams2009dynamics}. Onboard estimation must fuse noisy IMU, GPS, and barometer via nonlinear filters~\cite{sola2017quaternion}, while wind introduces unmodeled forces~\cite{moorhouse1982us}. The robustness of geometric cooperative controllers to these conditions remains largely uncharacterized.

\subsection{Design Philosophy: Hazard-Oriented Decomposition}

A central principle in safety-critical system engineering is the early identification and systematic mitigation of dominant hazards~\cite{STPAleveson}. Rather than designing a monolithic controller and verifying safety post-hoc, each GPAC layer directly addresses a specific cooperative transport hazard: anti-swing regulation stabilizes pendular modes, adaptive estimation mitigates load uncertainty, the ESO compensates environmental disturbances, and the CBF layer enforces physical constraints as a runtime supervisor. The hierarchical multi-rate structure with deliberate timescale separation limits fault propagation between layers; the fully decentralized implementation eliminates single points of failure~\cite{koopman2017autonomous}. Table~\ref{tab:failure_modes} in Section~\ref{sec:results} validates this mapping with measured performance data.

\subsection{Contributions}

The GPAC architecture decomposes the cooperative transport problem into $N$ identical single-agent subproblems via a key insight: each drone independently estimates its payload mass share $\hat{\theta}_i \approx m_L/N$ from local cable measurements, and the combined forces automatically sum to the correct total:
\begin{equation}
  \sum_{i=1}^{N} F_i = \sum_{i=1}^{N} \hat{\theta}_i \cdot u \;\longrightarrow\; m_L \cdot u,
  \label{eq:force_convergence}
\end{equation}
where $u := g\,e_3 + \ddot{p}_L^d$ is the common payload acceleration demand (gravity plus desired payload acceleration), which every agent computes identically from the shared reference trajectory. This implicit coordination property---which does not hold for linearized or impedance-based approaches---eliminates the need for a centralized coordinator, payload or cable state exchange, consensus protocols, or knowledge of $N$ and $m_L$. The only inter-agent information is a low-rate GPS position broadcast (10\,Hz) used for collision avoidance. The specific contributions are:
\begin{enumerate}[leftmargin=*, itemsep=2pt]
  \item \textbf{Decentralized geometric control on $\SEthree\times(\Sph^2)^N$.} To our knowledge, the first decentralized controller for multi-UAV cable transport that operates directly on the nonlinear manifold without linearization. Each agent's policy depends only on its own state and local cable measurements; the only shared information is a reference trajectory (broadcast before flight) and neighbor GPS positions (10\,Hz, for collision avoidance only). No payload state, cable state, or adaptive parameter is exchanged.

  \item \textbf{Concurrent learning adaptive estimation without PE.} A decentralized adaptive law in which each drone estimates $\hat{\theta}_i \to m_L/N$ from local cable tension and direction, with exponential convergence guaranteed by a rank-maximizing history buffer---even during near-static hover where classical adaptive laws fail.

  \item \textbf{CBF safety filter with formal geometric compatibility.} A modular safety layer enforcing cable tautness, angle, tilt, swing rate, and collision constraints with input-to-state safety (ISSf) guarantees. Theorem~\ref{thm:compatibility} proves that the CBF-induced force modifications preserve the almost-global exponential stability of the geometric attitude controller on $\SOthree$.

  \item \textbf{Sensor-realistic validation with flexible cable dynamics.} High-fidelity Drake-based simulation with bead-chain cables (capturing slack-to-taut transitions and wave propagation), a 15-state ESKF fusing noisy IMU/GPS/barometer, and Dryden wind turbulence, achieving $23.5 \pm 1.5$\,cm payload tracking RMSE over a $6 \times 3$\,m workspace (3.3\% of workspace diagonal) at under 1\,MFLOP/s per agent.
\end{enumerate}

These contributions demonstrate that geometric rigor, operational decentralization, and safety guarantees can coexist in a single architecture. The remainder of this paper presents the system model (Section~\ref{sec:modeling}), control architecture (Section~\ref{sec:control}), adaptive estimation (Section~\ref{sec:estimation}), CBF safety filter (Section~\ref{sec:safety}), simulation results (Section~\ref{sec:results}), and conclusion (Section~\ref{sec:conclusion}).
