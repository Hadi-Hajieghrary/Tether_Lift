
The world frame~$\mathcal{W}$ has $e_3$ aligned upward. Rotations lie in $\SOthree$; the hat map $(\cdot)^{\wedge}:\R^3 \to \sothree$ satisfies $\hatmap{v}w = v \times w$. Cable directions $q_i \in \Sph^2$ have tangent-space projection $P(q) = I_3 - qq^\top$. The full system evolves on
\begin{equation}
  \mathcal{Q} = \underbrace{\SEthree}_{\text{payload}} \times \prod_{i=1}^{N}\!\underbrace{\SEthree \times \Sph^2}_{\text{drone }i\text{ + cable}}, \quad \dim(\mathcal{Q}) = 6 + 8N.
  \label{eq:manifold}
\end{equation}
The singularity-free geometric error functions are:
\begin{align}
  e_{R_i} &= \tfrac{1}{2}(R_{d_i}^\top R_i - R_i^\top R_{d_i})^\vee, \label{eq:eR}\\
  \Psi_{q_i} &= 1 - q_{d_i} \cdot q_i \in [0,2], \label{eq:psi_q}\\
  e_{q_i} &= P(q_i)\,q_{d_i} \in T_{q_i}\Sph^2, \label{eq:eq}
\end{align}
where $e_{q_i}$ is the negative gradient of $\Psi_{q_i}$ restricted to $T_{q_i}\Sph^2$.

\subsection{Quadrotor and Payload Dynamics}

Each of $N$~identical quadrotors (mass $m_Q$, body-frame inertia $J$) obeys
\begin{align}
  m_Q \ddot{p}_i &= m_Q g\, e_3 + f_i R_i e_3 + F_i^{\text{cable}} + F_i^{\text{wind}}, \label{eq:quad_trans}\\
  J\dot{\Omega}_i &= -\Omega_i \times J\Omega_i + \tau_i + \tau_i^{\text{ext}}, \label{eq:quad_rot}
\end{align}
with kinematics $\dot{R}_i = R_i \hatmap{\Omega}_i$, where $f_i$ is the scalar thrust magnitude, $\tau_i$ is the body-frame control torque, $F_i^{\text{cable}}$ is the cable tension force, and $F_i^{\text{wind}}$ is the aerodynamic disturbance. Gravity acts as $-ge_3$ with the thrust $f_i R_i e_3$ directed along the body $z$-axis. The payload (mass $m_L$, position $p_L$) satisfies
\begin{equation}
  m_L \ddot{p}_L = -m_L g\, e_3 + \textstyle\sum_{i=1}^{N} F_i^L + F^{\text{contact}} + F_L^{\text{wind}},
  \label{eq:payload}
\end{equation}
where $F_i^L$ is the cable force from the $i$-th rope at the payload attachment point, and $F^{\text{contact}}$ denotes the ground normal and Coulomb friction forces.

The cable direction $q_i \in \Sph^2$ from the payload toward the quadrotor evolves as $\dot{q}_i = \omega_{q_i} \times q_i$ with $\omega_{q_i} \in T_{q_i}\Sph^2$. Under the constraint $p_i = p_L + R_L\rho_i^L + L_i q_i$ (where $\rho_i^L$ is the attachment offset in the payload body frame), the swing dynamics on $\Sph^2$ are governed by the projected quadrotor acceleration and gravitational restoring torque.

\subsection{Bead-Chain Cable Model}

Each cable is discretized as $n_b$ point-mass beads connected by $n_b+1$ tension-only spring-damper segments. Let $b_0$ denote the quadrotor attachment, $b_1,\ldots,b_{n_b}$ the bead positions, and $b_{n_b+1}$ the payload attachment. Each segment has rest length $L_0 = L_{\text{rest}}/(n_b+1)$, stretch $\Delta_j = \norm{b_{j-1}-b_j} - L_0$, and unit direction $\hat{e}_j = (b_{j-1}-b_j)/\norm{b_{j-1}-b_j}$. The tension-only force law is
\begin{equation}
  T_j = \begin{cases} k_s \Delta_j + c_s [\dot{\Delta}_j]^+, & \Delta_j > 0,\\0, & \Delta_j \leq 0, \end{cases}
  \label{eq:cable_tension}
\end{equation}
where $[\cdot]^+ = \max(\cdot,0)$ restricts damping to the stretching phase. The segment stiffness is derived from a maximum-stretch design criterion ($\epsilon_{\max} = 15\%$):
\begin{equation}
  k_s = \frac{F_{\text{load}}}{L_{\text{rest}}\,\epsilon_{\max}} \cdot (n_b + 1), \quad c_s = c_{\text{ref}}\sqrt{k_s/k_{\text{ref}}},
  \label{eq:stiffness}
\end{equation}
with $F_{\text{load}} = m_L g/N$ and reference values $k_{\text{ref}} = 300$\,N/m, $c_{\text{ref}} = 15$\,N$\cdot$s/m. Each bead (mass $m_b = m_{\text{rope}}/n_b$, with $m_{\text{rope}} = 0.2$\,kg) obeys $m_b\ddot{b}_j = F_j - F_{j+1} - m_bg\,e_3$. This model naturally captures wave propagation (${\sim}\sqrt{k_s L_0/m_b}$), distributed inertia, and the impulsive loading during slack-to-taut transitions absent from rigid-link models. Rope rest lengths are sampled from $L_{\text{rest},i} \sim \mathcal{N}(\bar{L}_i, \sigma_{L_i}^2)$ to model manufacturing uncertainty, yielding up to 19\% asymmetry in the simulation.

\subsection{Sensor Models and Wind Disturbance}

Each quadrotor carries an onboard sensor suite. The \emph{IMU} (200\,Hz) provides body-frame specific force $\tilde{a}_i = R_i^\top(\ddot{p}_i + ge_3) + b_{a_i} + n_{a_i}$ and angular velocity $\tilde{\omega}_i = \Omega_i + b_{g_i} + n_{g_i}$ with noise densities $\sigma_a = 0.004$\,m/s$^2$/$\sqrt{\text{Hz}}$, $\sigma_g = 5\times 10^{-4}$\,rad/s/$\sqrt{\text{Hz}}$, and Gauss--Markov biases ($\dot{b} = -b/\tau + \eta$, $\tau = 3600$\,s). The \emph{GPS receiver} (10\,Hz) provides position with $\sigma_{xy} = 0.02$\,m, $\sigma_z = 0.05$\,m, and configurable dropout probability. The \emph{barometric altimeter} (25\,Hz) has a three-component noise model: white noise ($\sigma_w = 0.3$\,m), correlated noise ($\tau_c = 5$\,s), and slow bias drift (0.002\,m/s), with 0.1\,m quantization. Cable tension $T_i$ is measured via a load cell ($\sigma_T = 0.5$\,N) and cable direction $q_i$ via a 2-axis encoder ($\sigma_q = 0.02$\,rad).

Wind follows the Dryden turbulence spectrum (MIL-F-8785C)~\cite{moorhouse1982us} with forming filter $H_\alpha(s) = \sigma_\alpha\sqrt{2V/L_\alpha}/(s + V/L_\alpha)$, turbulence intensities $\sigma_u = \sigma_v = 0.5$\,m/s, $\sigma_w = 0.25$\,m/s, and altitude-dependent scaling $\sigma_\alpha(h) \propto (h/h_{\text{ref}})^{1/6}$. Spatial correlation between agents decays as $\rho(p_i,p_j) = \exp(-\norm{p_i - p_j}/\ell_c)$ with $\ell_c = 10$\,m, ensuring that closely spaced drones experience similar wind fields. Table~\ref{tab:params} collects all key parameters.

\begin{table}[t]
  \centering
  \caption{Key System Parameters}
  \label{tab:params}
  \begin{tabular}{@{}lcc@{}}
    \toprule
    \textbf{Parameter} & \textbf{Symbol} & \textbf{Value} \\
    \midrule
    Quadrotor mass / count & $m_Q$ / $N$ & 1.5\,kg / 3 \\
    Payload mass / radius  & $m_L$ / $r_L$ & 3.0\,kg / 0.15\,m \\
    Formation radius       & $r_f$ & 0.6\,m \\
    Beads per cable        & $n_b$ & 8 \\
    Rope mass / max stretch & $m_{\text{rope}}$ / $\epsilon_{\max}$ & 0.2\,kg / 15\% \\
    IMU rate / noise       & $f_{\text{IMU}}$ / $\sigma_a$ & 200\,Hz / $0.004$\,m/s$^2$/$\!\sqrt{\text{Hz}}$ \\
    GPS rate / noise       & $f_{\text{GPS}}$ / $\sigma_{xy}$ & 10\,Hz / 0.02\,m \\
    Baro rate / noise      & $f_{\text{baro}}$ / $\sigma_w$ & 25\,Hz / 0.3\,m \\
    Wind intensity         & $\sigma_u, \sigma_v$ / $\sigma_w$ & 0.5 / 0.25\,m/s \\
    Simulation step        & $\Delta t_{\text{sim}}$ & 0.2\,ms \\
    \bottomrule
  \end{tabular}
\end{table}
