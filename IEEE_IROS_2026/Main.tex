\documentclass[letterpaper, 10 pt, conference]{./ieeeconf} 


\let\labelindent\relax
\IEEEoverridecommandlockouts 
\overrideIEEEmargins 
\usepackage{pifont}
\usepackage{mathtools}
\usepackage{caption} 
\usepackage{cite}
\usepackage{times}
\usepackage[pdftex]{graphicx}
\graphicspath{{./Figures/}}
\usepackage{subfiles}

\DeclareGraphicsExtensions{.pdf,.jpeg,.png,.jpg}
\usepackage{subcaption}
\usepackage{epstopdf}
\usepackage{amsmath}
\usepackage{amssymb}
\usepackage{amsfonts}
\usepackage{mathrsfs}
% \usepackage{amsthm}
\newtheorem{assumption}{Assumption}
\usepackage{tabularx}
\usepackage{booktabs}
\usepackage{longtable}
\usepackage{tikz}
\usetikzlibrary{positioning, arrows.meta, calc, fit}
\usepackage{array}
\newcolumntype{P}[1]{>{\raggedright\arraybackslash}p{#1}}

\usepackage{enumitem}
\usepackage{siunitx}
\usepackage{ragged2e}
\usepackage{float}
\usepackage{algorithm}
\usepackage{algorithmic}
% subfiles already loaded above
\usepackage{titlesec}
\usepackage{hyperref}
\titlespacing{\section}{0pt}{3pt}{2pt}
\titlespacing{\subsection}{0pt}{0pt}{0pt}

% ---------- column helpers ----------
\newcolumntype{C}[1]{>{\centering\arraybackslash}m{#1}}
\newcolumntype{L}[1]{>{\RaggedRight\arraybackslash}m{#1}}
% ------------------------------------

\newcommand{\R}{\mathbb{R}}
\newcommand{\Sph}{\mathbb{S}}
\newcommand{\SOthree}{\mathrm{SO}(3)}
\newcommand{\SEthree}{\mathrm{SE}(3)}
\newcommand{\sothree}{\mathfrak{so}(3)}
\newcommand{\norm}[1]{\left\lVert #1 \right\rVert}
\newcommand{\abs}[1]{\left| #1 \right|}
\newcommand{\hatmap}[1]{\widehat{#1}}
\DeclareMathOperator{\tr}{tr}
\DeclareMathOperator{\diag}{diag}

% Math operators
\DeclareMathOperator*{\argmin}{arg\,min}
\DeclareMathOperator*{\argmax}{arg\,max}

\newcommand{\Rnonneg}{\mathbb{R}_{\ge 0}}
\newcommand{\Prob}{P}
\newcommand{\ind}{\mathbb{I}} 


\newcounter{example}
\newcommand{\example}[1]{%
 \stepcounter{example}%
 \par\vspace{12pt}%
 \textbf{Example \theexample: #1}%
 \par\vspace{6pt}%
 \setcounter{subsection}{0}
}

\sisetup{
  round-mode          = places,
  round-precision     = 2,
}


\newtheorem{theorem}{Theorem}[section]
\newtheorem{definition}[theorem]{Definition}
\newtheorem{lemma}[theorem]{Lemma}
\newtheorem{corollary}[theorem]{Corollary}
\newtheorem{proposition}{Proposition}

\newtheorem{remark}[theorem]{Remark}

\setlength{\textfloatsep}{5pt plus 2pt minus 4pt} % Adjust space between float and text
\setlength{\belowcaptionskip}{-5pt} % Reduce space after the caption


\hyphenation{op-tical net-works semi-conduc-tor}

\setlength{\abovedisplayskip}{5pt}
\setlength{\belowdisplayskip}{5pt}

\begin{document}
\title{Cooperative Transport via Flexible Cable Suspensions:\\ Decentralized Geometric Control with Adaptive Estimation and Safety Guarantees}
\author{Hadi Hajieghrary$^{1}$, Benedikt Walter$^{2}$, Miguel Hurtado$^{3}$, and Paul Schmitt$^{4}$% <-this % stops a space
\thanks{$^{1}$Hadi Hajieghrary \{{\tt\small Hadi.Hajieghrary@gatech.edu}\}, ...}%
 }


\maketitle

\begin{abstract}
We present the Geometric Position and Attitude Control (GPAC) architecture for decentralized cooperative aerial transport via flexible cable suspensions. Each of $N$~quadrotors operates with zero knowledge of the team size~$N$, payload mass~$m_L$, or any other agent's state, yet the closed-loop system maintains geometric control on the full $\mathrm{SE}(3) \times (\mathbb{S}^2)^N$ configuration manifold with Lyapunov-certifiable stability guarantees. A concurrent learning adaptive law enables each drone to independently estimate its payload share $\hat{\theta}_i \to m_L/N$ without persistent excitation, and a modular Control Barrier Function safety filter enforces cable tautness, swing angle, tilt, and collision constraints with input-to-state safety (ISSf) guarantees---bounding constraint violations under disturbances---provably compatible with the geometric attitude controller. The framework is validated in a high-fidelity Drake-based simulation incorporating flexible bead-chain cable dynamics, onboard sensor models (IMU, GPS, barometer) with Error-State Kalman Filter fusion, and Dryden wind turbulence. The decentralized architecture achieves 22.9\,cm payload tracking RMSE, maintains safety constraints within ISSf margins, and requires less than 1\,MFLOP/s per agent, demonstrating feasibility for embedded deployment.
\end{abstract}

\section{Introduction}
\label{Sec:Introduction}
\subfile{./Sections/Section_I_Introduction}

\section{System Modeling}
\label{Sec:System_Modeling}
\label{sec:modeling}
\subfile{./Sections/Section_II_System_Modeling}

\section{GPAC Control Architecture}
\label{Sec:GPAC_Control_Architecture}
\label{sec:control}
\subfile{./Sections/Section_III_GPAC_Control_Architecture}

\section{Decentralized Adaptive Estimation}
\label{Sec:Decentralized_Adaptive_Estimation}
\label{sec:estimation}
\subfile{./Sections/Section_IV_Decentralized_Adaptive_Estimation}

\section{CBF Safety Filter}
\label{Sec:CBF_Safety_Filter}
\label{sec:safety}
\subfile{./Sections/Section_V_CBF_Safety_Filter}

\section{Simulation Environment}
\label{Sec:Simulation_Environment}
\label{sec:simulation}
\subfile{./Sections/Section_VI_Simulation_Environment}

\section{Results}
\label{Sec:Results}
\label{sec:results}
\subfile{./Sections/Section_VII_Results}

\section{Conclusion}
\label{Sec:Conclusion}
\label{sec:conclusion}
\subfile{./Sections/Section_VIII_Conclusion}

\bibliographystyle{IEEEtran}
\bibliography{./References/References.bib}



\end{document}